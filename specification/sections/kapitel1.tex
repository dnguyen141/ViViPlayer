\section{Mission des Projekts}

\subsection{Erläuterung des zu lösenden Problems}
Das Projekt ViViPlayer (Visionsvideo Player) ist eine Webanwendung, welche die gemeinsame Arbeit 
bei der Anforderungsanalyse im Requirement Engineering effektiver und kommunikativer macht.
Damit ist es möglich sich gemeinsam ein Visionsvideo anzuschauen und direkt in der Anwendung User Stories
zu verfassen. Durch interaktive Funktionen kann man sich einen Überblick schaffen über die Ansichten und Vorstellungen der Teilnehmer.

Die Anwendung verfolgt also das Ziel effektiv Anforderungen zu sammeln und dabei alle möglichst viel mit einzubeziehen.
Deswegen ist dieses Projekt als Webanwendung konzipiert, um möglichst vielen einfachen Zugriff auf das Programm zu geben.

\subsection{Wünsche und Prioritäten des Kunden}
	Es folgen die Wünsche des Kunden, nach absteigender Priorität geordnet:
	\begin{itemize}
		\item In der Webanwendung ist es möglich synchron mit den anderen Teilnehmern Video wiederzugeben. Der Moderator ist hierbei derjenige der den Player bedient.
		\item Der Moderator kann entweder das Video bis zum Ende abspielen oder bis zum nächsten Shot.
		\item Das Video wird automatisch in Shots segmentiert, sodass der Moderator an die verschiedenen Punkte springen kann. Der Moderator kann manuell segmentieren.
		\item Die Teilnehmer können User Stories zu einem Shot oder mehreren Shots direkt in der Webanwendung erstellen mit Hilfe einer Schablone.
		\item Die Teilnehmer können sich vertraulich und sicher mit einer TAN anmelden.
		\item User Stories können direkt als ``.csv''-Dateien mit zusätzlichen Bildern exportiert werden. Man kann diese User Stories direkt in Trello leicht importieren.
		\item Es ist möglich direkt Verständnisfragen und Umfragen zu stellen.
		\item Annotationen können direkt im Video eingeblendet werden.
		\item User Stories, Umfragen, Verständnisfragen, Screenshots, Kommentare werden am Ende der Sitzung zu den Shots zugeordnet und in eine ``.odt''-Datei exportiert.
		\item Die Webanwendung kann automatisch vervollständigen oder Rückmeldungen anhand von Anleitungen geben.
		\item Implementierung anhand von Design Patterns zur späteren Erweiterung als Android App zum Beispiel.
	\end{itemize}

\subsection{Domänenbeschreibung}
    Eingesetzt wird dieses Programm in einem Treffen mit einem Kunden zur Anforderungsanalyse.
    Hierbei kann sich sowohl Online als auch in Präsenz getroffen werden. Auch eine hybride Variante,
    wo sowohl Online- als auch Präsenzteilnehmer anwesend sind ist denkbar. 
    Anwesend sind immer ein bis zwei Moderatoren, die die Sitzung leiten und den Videoplayer bedienen.
	Zielgruppe sind also sowohl die Entwickler als auch der Vertreter des Kunden, welcher auch verantwortlich für die Anforderungen ist.
\subsection{Maßnahmen zur Anforderungsanalyse}
    Um die Anforderungen auf unserer Seite zu klären haben wir mehrere Prototypen gebaut.
    Direkt zu Anfang haben wir einen Prototypen erstellt, um zu ermitteln wie man Video segmentieren kann.
    Zudem haben wir einen weiteren Prototypen erstellt, mit welchem wir die Synchronisation erprobt haben.
    Hierbei stand rein die technische Umsetzung im Vordergrund und wie man Videos synchronisieren kann.
    Beide Prototypen sind JavaScript Anwendungen die bereits im Browser laufen.

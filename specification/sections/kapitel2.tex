\section{Rahmenbedingungen und Umfeld}

\subsection{Einschränkungen und Vorgaben}
Der Zugang zu der Web-App ist durch Anmeldedaten beschränkt, d.h. sind Username und Passwort notwendig, um sich einzuloggen. Alternative Login wie TAN ist auch möglich. Nutzer kann entweder als Moderator einloggen oder als Teilnehmer. Moderator und Teilnehmer können in dieser Web-App gleichzeitig eine interaktive Informationsaustausch durchführen, also eine Session. Eine Session kann allerdings nur von dem Moderator vorbereitet werden.\linebreak
\linebreak
Bevor ein Session anfängt, hat der Moderator die Möglichkeiten, Video von Cloud einzubetten, das Video abzuspielen und auch zu segmentieren. Nachdem Hochladen konnte der Moderator das Video entweder automatisch oder manuell segmentieren. Noch ein weiteres Feature ist Annotationen für Verständnisfragen in dem Video zu legen. Werden keine weitere Änderungen für das Video gemacht, kann Moderator auf ''Fertig'' klicken, dann startet eine Session.\linebreak
\linebreak
Während des Sessions kann der Moderator das Video abspielen und navigieren für den Teilnehmer. Teilnehmer kann das Abspielen des Videos live verfolgen, Fragen beantworten, und auch Stories schreiben. Würde eine Frage im bestimmten Zeitpunkt kommen, kann der Teilnehmer auf die Antwort einfach klicken, und die wird dann gleich in Form eines Balken- oder Kreisdiagramms sichtbar. Der Teilnehmer kann auch eine alternative Antwortmethode wählen, falls eine Umfrage nicht bevorzugt wird. Alle Annotationen und Antworten werden als odt-Datei zusammengefasst, und im Server gespeichert. Zum Schreiben der User Stories kann auch der Teilnehmer ein Screenshot hinfügen mit Copy und Paste. Schlusselsätze als odt-Datei zusammen mit dem Screenshot werden zu JIRA Board exportiert.\linebreak
\linebreak
Nach Kundenpräferenz ist die Anzahl von Moderator und Teilnehmer 18 Personen in einem Session.

\subsection{Anwender}
Die Web-App ist für Softwareentwickler und auch Kunden, die Kenntnise über Requirements Engineering besetzen, geeignet. Was gemeint hier bei Kunden ist, diejenigen, den das Projekt gehört. In einem Session wird Softwareentwickler Moderator, und die Kunden werden Teilnehmer sein. Die Web-App sollte von Nutzer mit durchschnittlichen Computerkenntnisse bedienbar sein, damit sie sich leicht durch die Web-App navigieren lassen.

\subsection{Schnittstellen und angrenzende Systeme} 
In diesem Projekt wird Django als Framework benutzt, da Django sowohl Frontend als auch Backend übernimmt. HTML5, CSS, JavaScript können in Django integriert werden mithilfe von frontend Bibliotheken von Django. Für Backend wird SQLite als Datenbank benutzt.
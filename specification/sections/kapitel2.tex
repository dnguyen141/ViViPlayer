\section{Rahmenbedingungen und Umfeld}

\subsection{Einschränkungen und Vorgaben}
Diese Web-App sollte hauptsachlich für Desktop Browser verfügbar sein. Im Hauptmenu beim ersten Öffnen sollte der Nutzer in der Lage sein, auszuwählen, entweder er sich als Moderator einloggen, oder eine ViViPlayer-Session als Teilnehmer teilnimmt. Eine ViViPlayer-Session zwischen Moderator und Teilnehmer kann jedoch nur von dem Moderator vorbereitet werden.\linebreak
\linebreak
Bevor eine Session anfängt, hat der Moderator die Möglichkeiten, ein Video von Cloud einzubetten, das Video abzuspielen und auch zu segmentieren. Der Moderator kann 
das Video entweder automatisch oder manuell segmentieren. Multiple-Choice-Fragen können in Form von Annotationen in dem Video eingebettet werden. Der Moderator kann die ViViPlayer-Session starten, wenn keine weiteren Änderungen gemacht werden.\linebreak
\linebreak
Während der Session kann der Moderator das Video abspielen und für den Teilnehmer navigieren. Moderator muss zuerst eine Session-Code ausstellen und an Teilnehmer verteilen, damit Teilnehmer die ViViPlayer-Session teilnehmen kann. Der Teilnehmer kann danach in der Session das Abspielen des Videos live verfolgen, Fragen und auch Umfragen beantworten. Der Teilnehmer kann auf die Annotation zum Antworten einfach klicken, und die wird dann gleich als Balkendiagramm sichtbar. 
Sowohl Moderator als auch Teilnehmer können User Story herstellen während das Abspielen des Videos. Zum Schreiben der User Stories ist es auch möglich ein Screenshot aufzunehmen 
mithilfe von PySceneDetect ohne das Bild manuell hochzuladen. Schlusselsätze von User Story zusammen mit dem Screenshot können als CSV Datei und Bilddatei exportiert werden, damit Nutzer zu TRELLO-Board importieren kann. 
Alle Annotationen, geschriebene User Stories und Antworten können zusammengefasst und als odt-Datei exportiert werden.\linebreak
\linebreak
Nach Kundenpräferenz ist die Anzahl von Nutzer der Web-App unbegrenzt. Zunächst wird die Web-App 18 Personen bedient.

\subsection{Anwender}
Sowohl Moderator als auch Teilnehmer dieser Web-App sollten mindestens Kenntnise über Requirements Engineering besitzen. Die Rolle Moderator sollte von entweder SCRUM Master 
oder Projektleiter behandelt werden. Kundenvertreter oder Product Owner übernimmt die Rolle als Teilnehmer, weil sie ihre Vision dann durch User Story übermitteln können. 
Diese Web-App sollte leicht bedientbar sein, damit Nutzer mit durchschnittliche Computerkenntnise durch die App leicht navigieren können.

\subsection{Schnittstellen und angrenzende Systeme} 

In diesem Projekt wird Django als Framework benutzt, da Django sowohl Frontend als auch Backend übernimmt. HTML5, CSS, JavaScript können in Django integriert werden mithilfe 
von Frontend-Bibliotheken von Django. Für Backend wird SQLite als Datenbank benutzt. Direkte Aufnahme von Screenshots wird mithilfe von einem internen Programm PySceneDetect 
und OpenCV ermöglicht.

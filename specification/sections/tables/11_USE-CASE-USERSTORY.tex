\begin{tabularx}{\linewidth}{|l|X|}
	\hline
	Use Case Nr. 11			&  \textbf{ViViPlayer User-Story Schreiben} \\ \hline
	Erläuterungen			&  Systembediener möchte eine User Story schreiben und klickt in 
							   das Textfeld. \\ \hline
	Status					&  Der Nutzer befindet sich auf der Hauptseite der ViViPlayer App 
							   \\ \hline
	Systemgrenzen (Scope)	&  User-Story-System. \\ \hline
	Ebene					&  Hauptfunktion. \\ \hline
	Vorbedingung			&  Das System ist betriebsbereit. Der Benutzer hat Moderator- oder 
							   Teilnehmer-Rolle, der befindet sich in einer ViViPlayer-Session. \\ \hline
	Mindestgarantie			&  Der Benutzer erhält eine Fehlermeldung und es werden keine Daten 
							   in dem Server abgespeichert. Die Daten von anderen Benutzer werden trotzdem abgespeichert. \\ \hline
	Erfolgsgarantie			&  Der Benutzer hat eine User Story verfasst und der Server hat
							   diese abgespeichert. Die User Story wurde einem Shot
							   zugeordnet.\\ \hline
	Stakeholder				&  Systembediener - möchte User Stories einfach verfassen können.\\ 
                            &  Entwickler - möchten User Stories für ihr Projekt haben. \\ 
                               \hline
	Hauptakteur				&  Systembediener mit Moderator- oder Teilnehmer-Rolle. \\ \hline
	Auslöser				&  Der Benutzer wählt den User Story Modus aus. \\ \hline	
	Hauptszenario			&  1. Der Benutzer wählt den User Story Modus aus. \\
                            &  2. Der Benutzer gibt seine User Story ein und bestätigt, wenn er 
                               fertig ist. \\
							&  3. Das System verarbeitet diese Anfrage und speichert die User 
							   Story ab. Zudem wird sie dem jeweiligen Shot zugeordnet. \\
							&  4. Das System zeigt die neue User Story den anderen Nutzern an.\\
							&  5. Der Nutzer kann eine weitere User Story verfassen. \\ \hline
	Erweiterungen			& 3a. WENN ein Fehler auftritt DANN wird der Benutzer 
	                          benachrichtigt und er kann die User Story nochmal eingeben. Es werden keine Daten in der Datenbank abgespeichert. \\ \hline
	Priorität				&  hoch \\ \hline
	Verwendungshäufigkeit	&  häufig \\ \hline
\end{tabularx}

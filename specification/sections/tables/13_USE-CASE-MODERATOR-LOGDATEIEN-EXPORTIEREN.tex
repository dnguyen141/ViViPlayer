\begin{tabularx}{\linewidth}{|l|X|}
	\hline
	Use Case Nr. 12			& \textbf{ViViPlayer Moderator Log-Dateien Exportieren} \\ \hline
	Erläuterungen			& Nach dem Beendigung der ViViPlayer-Session sollte einige Dateien
							  als Berichte exportiert werden.\\ \hline
	Status					& Der Benutzer befindet sich auf der ViViPlayer-Seite der 
							  Web-App. Das Meeting ist schon zu Ende und als Moderator möchte er die ViViPlayer-Session beenden. \\ \hline
	Systemgrenzen (Scope)	& Gesamtsystem. \\ \hline
	Ebene					& Hauptfunktion. \\ \hline
	Vorbedingung			& Das System ist betriebsbereit.\\ \hline
	Mindestgarantie			& Die Log-Dateien werden mit einer Fehlermeldung exportiert. Die 
							  Daten von den Benutzer, die nicht Fehler haben, werden exportiert. Das Dateisystem vom Server funktioniert einwandfrei. \\ \hline
	Erfolgsfall				& Die exportierten Dateien wurden im Server abgespeichert. Der 
							  Benutzer konnte solche Dateien problemlos herunterladen. \\ \hline
	Stakeholder				& Systembediener - möchte Log-Dateien exportieren können.\\ 
							& Entwickler - möchten die Log-Dateien zu Trello- oder Jira-Board 
							  importieren können. \\ \hline
	Hauptakteur				& Systembediener mit Moderator-Rolle \\ \hline
	Auslöser				& Der Benutzer beendet die ViViPlayer-Session. \\ \hline	
	Hauptszenario			& 1. Der Benutzer beendet die ViViPlayer-Session. \\
							& 2. Das System bekommt das Beenden-Signal, und versucht, User 
							  Stories und Kommentare von allen Benutzer sowie Screenshot des ViVis abzuspeichern. \\
							& 3. Der Benutzer kann sehen, dass der Exportierung-Prozess 
							  fertig ist und klickt auf ``Exportieren''. \\
							& 4. Das System erzeugt eine URL zum Herunterladen. \\
							& 5. Der Benutzer speichert die Dateien in seinem System ab. \\ 
							  \hline
	Erweiterungen			& 2a. WENN ein Fehler auftritt DANN wird die Log-Dateien mit 
							  Fehlermeldung im Server-Dateisystem abgespeichert. Der Benutzer bekommt dann keine URL zum Herunterladen. \\ \hline
	Priorität				& sehr hoch \\ \hline
	Verwendungshäufigkeit	& regelmäßig \\ \hline
\end{tabularx}


\begin{tabularx}{\linewidth}{|l|X|}
	\hline
	Use Case Nr. 09			& \textbf{Um-/Verständnisfrage in Session stellen} \\ \hline
	Erläuterungen			& Als Moderator kann der Benutzer mehrere Um-/Verständnisfragen 
							  während der Session erstellen, um Feedbacks von Teilnehmer zu bekommen. \\ \hline
	Systemgrenzen (Scope)	& Gesamtsystem. \\ \hline
	Ebene					& Hauptfunktion. \\ \hline
	Vorbedingung			& Die Web-App ist betriebsbereit. Der Benutzer hat Moderator-
							  Rechten. Er ist in einer Session mit anderen Benutzern. \\ \hline
	Mindestgarantie			& Keine Umfrage wird erstellt. \\ \hline
	Erfolgsfall 			& Die Um-/Verständnisfrage, die 30 Sekunde dauert, wurde erstellt, 
							  alle Teilnehmer konnten die sehen und interagieren. \\ \hline
	Stakeholder				& Systembediener (Moderator) - will eine produktive Session mit den 
							  Teilnehmer via ViViPlayer. \\
							& Systembesizer (Systemadministrator) - möchte alle Sessions in der 
							  Web-App erfolgreich durchgeführt werden. \\ \hline
	Hauptakteur				& Systembediener (Moderator). \\ \hline
	Auslöser				& Der Benutzer klickt auf ``Frage''-Tab. \\ \hline	
	Hauptszenario			& 1. Der Benutzer klickt auf ``Frage''-Tab. \\
							& 2. Das System zeigt ein neues Eingabe-Feld an. \\ 
							& 3. Der Benutzer gibt den Frage-Texte und den Typ der Frage (Um- 
							  oder Verständnisfrage) für die ein und bestätigt. \\
							& 4. Das System zeigt die Um-/Verständnisfrage in 30 Sekunden für 
							  alle Teilnehmer an. \\ \hline
	Erweiterungen			& 4a. WENN es keine Frage-Texte oder keine Auswahl für die 
							  Um-/Verständnisfrage gibt, DANN zeigt das System eine Fehlermeldung an. Zurück zu Schritt 2. \\ \hline
	Priorität				& mittel \\ \hline
	Verwendungshäufigkeit	& weniger häufig \\ \hline
\end{tabularx}
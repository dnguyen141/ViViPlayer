\begin{tabularx}{\linewidth}{|l|X|}
	\hline
	Use Case Nr. 01			& \textbf{ViViPlayer Comments} \\ \hline
	Erläuterungen			&  Einen Kommentar/Satz verfassen. \\ \hline
	Status					&  Der Nutzer befindet sich auf der Hauptseite der ViViPlayer App. \\ \hline
	Systemgrenzen (Scope)	&  Kommentar-System\\ \hline
	Ebene					&  Hauptebene\\ \hline
	Vorbedingung			&  Das System ist betriebsbereit.\\ \hline
	Mindestgarantie			&  Der Benutzer erhält eine Fehlermeldung und es werden keine Daten in der Datenbank abgespeichert.\\ \hline
	Erfolgsgarantie			&  Der Benutzer verfasst eine Nachricht und der Server speichert diese ab. Die Nachricht wird einem Segment zugeordnet.\\ \hline
	Stakeholder				&  Benutzer - möchte Kommentare einfach verfassen können.\\ 
                            &  Kunden - möchten Feedback geben können. \\
                            &   Entwickler - möchten Kommentare für ihr Projekt haben und verfassen können. \\ \hline
	Hauptakteur				&  Der Benutzer\\ \hline
	Auslöser				&  Der Benutzer möchte eine User Story schreiben und klickt in das Textfeld. \\ \hline	
	Hauptszenario			&  1. Der Benutzer wählt den Kommentar Modus aus. \\
                            &  2. Der Benutzer gibt seinen Kommentar ein und bestätigt, wenn er fertig ist\\
							&  3. Das System verarbeitet diese Anfrage und speichert den Kommentar ab. Zudem wird er dem jeweiligen Segment zugeordnet. \\
							&  4. Das System zeigt den neuen Kommentar den anderen Nutzern an.\\
							&  5. Der Nutzer kann einen weiteren Kommentar verfassen. \\ \hline
	Erweiterung				& 3a. WENN ein Fehler auftritt DANN wird der Benutzer benachrichtigt und er kann den Kommentar nochmal eingeben. Es werden keine Daten in der Datenbank abgespeichert.\\ \hline
	Priorität				&  hoch \\ \hline
	Verwendungshäufigkeit	&  hoch \\ \hline
\end{tabularx}

\begin{tabularx}{\linewidth}{|l|X|}
	\hline
	Use Case Nr. 02			& \textbf{Moderator einloggen} \\ \hline
	Erläuterungen			& Der Benutzer möchte sich als Moderator einloggen, damit er eine 
							  Session im ViViPlayer abhalten kann. \\ \hline
	Systemgrenzen (Scope)	& Login-System \\ \hline
	Ebene					& Hauptfunktion \\ \hline
	Vorbedingung			& Die Web-App ist betriebsbereit. Der Benutzer befindet sich auf der
							  Hauptseite der Web-App. \\ \hline
	Mindestgarantie			& Das Einloggen des Benutzers wird abgebrochen, falls der eingegebene
							  Benutzername/die Email-Adresse nicht in der Datenbank oder das Passwort falsch ist. Eine Fehlermeldung wird dann ausgegeben.
							  \\ \hline
	Erfolgsfall 			& Der Zugriff des Benutzers wurde erfolgreich bestätigt. Der
							  Benutzer befindet sich auf der Moderator-Hauptseite. 
							  \\ \hline
	Stakeholder				& Systembediener (Moderator) - möchte die Funktionen der Web-App 
							  schnell wie möglich nutzen. \\
							& Systembesitzer (Systemadministrator) - möchte, dass die Funktionen 
							  der Web-App für die Benutzer mit korrekten Zugriffsrechten verfügbar sind.\\ \hline
	Hauptakteur				& Systembediener (Moderator) \\ \hline
	Auslöser				& Der Benutzer klickt auf die ``Einloggen als Moderator''-Option.\\ 
							  \hline	
	Hauptszenario			& 1. Der Benutzer klickt auf die ``Einloggen als 
							  Moderator''-Option.\\
							& 2. Das System zeigt das Moderator-Loginsformular an. \\
							& 3. Der Benutzer gibt sein Benutzernamee und sein Passwort ein. \\
							& 4. Das System validiert die vom Benutzer eingegebene Log-in-Daten. \\
							& 5. Der Benutzer wird im System angemeldet und weiter zur Moderator-Hauptseite 
							  geleitet. \\ \hline
	Erweiterungen			& 1a. WENN das System feststellt, dass die eingegebenen Daten
							  (Benutzername und Passwort) nicht gültig sind, DANN wird eine Fehlermeldung ausgegeben. Zurück zu Schritt 2. \\ \hline
	Priorität				& mittel \\ \hline
	Verwendungshäufigkeit	& häufig \\ \hline
\end{tabularx}

\begin{tabularx}{\linewidth}{|l|X|}
	\hline
	Use Case Nr. 02			& \textbf{Moderator Einloggen} \\ \hline
	Erläuterungen			& Als ein ScrumMaster/Projektleiter möchte der Benutzer einloggen und
							  damit er eine Session im ViViPlayer abhalten kann.  
							  \\ \hline
	Systemgrenzen (Scope)	& Login-System \\ \hline
	Ebene					& Hauptebene \\ \hline
	Vorbedingung			& Die Web-App ist betriebsbereit. Der Benutzer befindet sich in
							  Hauptseite der Web-App \\ \hline
	Mindestgarantie			& Das Log-in des Benutzers wird abgesagt, falls der eingegebene
							  Benutzername/die Email-Adresse nicht im Datenbank oder die Passwort falsch ist. Eine Fehlermeldung wird auf jedem Fall ausgegeben.
							  \\ \hline
	Erfolgsgarantie			& Der Zugriff des Benutzers wird erfolgreich bestätigt. Der
							  Benutzer wird zur Moderator-Hauptseiten weitergeleitet. 
							  \\ \hline
	Stakeholder				& Moderator - möchte die Funktionen der Web-App schnell 
							  wie möglich nutzen. \\
							& Herr Jianwei Shi - möchte die Funktionen der Web-App für die Benutzer 
							  mit korrektem Zugriffrecht  verfügbar sein.\\ \hline
	Hauptakteur				& Der Benutzer \\ \hline
	Auslöser				& Der Benutzer möchte als Moderator einloggen.
							  \\ \hline	
	Hauptszenario			& 1. Das System fordert den Benutzer auf, als Moderator ein
							  Benutzername/eine Email-Adresse sowie die Passwort einzugeben. \\
							& 2. Der Benutzer gibt sein Benutzername/seine Email-Adresse und seine
							  Passwort ein. \\
							& 3. Das System validiert die vom Benutzer eingegebene
							  Log-in-Daten. \\
							& 4. Dem Benutzer wird im System angemeldet und weiter zur
							  Moderator-Hauptseite geleitet. \\ \hline
	Erweiterung				& 3a. Wenn das System feststellt, dass die eingegebene Daten (Benutzername/
							  Email-Adresse und Passwort) nicht abgestimmt sind, wird eine Fehlermeldung 
							  ausgegeben. Der Benutzer wird dann zurück zur Hauptseite geleitet, damit
							  er noch einmal versuchen kann.
							  \\ \hline
	Priorität				& mittel \\ \hline
	Verwendungshäufigkeit	& häufig \\ \hline
\end{tabularx}

\begin{tabularx}{\linewidth}{|l|X|}
	\hline
	Use Case Nr. 02			& \textbf{Moderator Einloggen} \\ \hline
	Erläuterungen			& Der Benutzer möchte als Moderator einloggen, damit er eine 
							  Session im ViViPlayer abhalten kann. \\ \hline
	Systemgrenzen (Scope)	& Login-System \\ \hline
	Ebene					& Hauptfunktion \\ \hline
	Vorbedingung			& Die Web-App ist betriebsbereit. Der Benutzer befindet sich auf
							  Hauptseite der Web-App \\ \hline
	Mindestgarantie			& Das Einloggen des Benutzers wird abgesagt, falls der eingegebene
							  Benutzername/die Email-Adresse nicht im Datenbank oder die Passwort falsch ist. Eine Fehlermeldung wird dann ausgegeben.
							  \\ \hline
	Erfolgsfall 			& Der Zugriff des Benutzers wurde erfolgreich bestätigt. Der
							  Benutzer hat sich auf der Moderator-Hauptseite befindet. 
							  \\ \hline
	Stakeholder				& Systembediener - möchte die Funktionen der Web-App schnell 
							  wie möglich nutzen. \\
							& Herr Jianwei Shi (Systembesitzer) - möchte die Funktionen der 
							  Web-App für die Benutzer mit korrektem Zugriffsrecht verfügbar sein.\\ \hline
	Hauptakteur				& Systembediener, die Moderator-Rechte haben möchten \\ \hline
	Auslöser				& Der Benutzer möchte als Moderator einloggen. Er befindet sich 
							  auf der Hauptseite der Web-App, dann klickt auf die "Einloggen als Moderator"-Option.\\ \hline	
	Hauptszenario			& 1. Der Benutzer befindet sich auf der Hauptseite der Web-App, 
							  um einzuloggen. Er klickt auf die "Einloggen als Moderator"-Option.\\
							& 2. Das System fordert den Benutzer auf, als Moderator ein
							  Benutzername/eine Email-Adresse sowie die Passwort einzugeben. \\
							& 3. Der Benutzer gibt sein Benutzername/seine Email-Adresse und 
							  seine Passwort ein. \\
							& 4. Das System validiert die vom Benutzer eingegebene
							  Log-in-Daten. \\
							& 5. Der Benutzer wird im System angemeldet und weiter zur
							  Moderator-Hauptseite geleitet. \\ \hline
	Erweiterungen			& 1a. Wenn das System feststellt, dass die eingegebene Daten 
							  (Benutzername/Email-Adresse und Passwort) nicht abgestimmt sind, wird eine Fehlermeldung ausgegeben. Das System geht 
							  zurück zu Schritt 2. \\ \hline
	Priorität				& mittel \\ \hline
	Verwendungshäufigkeit	& häufig \\ \hline
\end{tabularx}

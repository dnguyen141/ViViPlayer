\begin{tabularx}{\linewidth}{|l|X|}
	\hline
	Use Case Nr. 07			& \textbf{Moderator Verständnisfragen Erstellen} \\ \hline
	Erläuterungen			& TAN erzeugen ist der dritte Schritt zur Vorbereitung einer 
							  ViViPlayer-Session. \\ \hline
	Systemgrenzen (Scope)	& Applikation. \\ \hline
	Ebene					& Hauptebene. \\ \hline
	Vorbedingung			& Die Web-App ist betriebsbereit. Der Benutzer hat Moderator
							  -Rechten. \\ \hline
	Mindestgarantie			& Es gibt keine Verständnisfragen für die Session.\\ \hline
	Erfolgsgarantie			& Eine Liste von Verständnisfragen ist erfolgreich hinzugefürgt.
							  \\ \hline
	Stakeholder				& Der Benutzer - will eine produktive Session mit den Teilnehmer 
							  via ViViPlayer. \\
							& Herr Jianwei Shi - möchte alle Sessions in der Web-App 
							  erfolgreich durchgeführt werden. \\ \hline
	Hauptakteur				& Der Benutzer. \\ \hline
	Auslöser				& Der Benutzer hat schon ein ViVi ausgewählt, segmentiert und 
							  klickt auf ``Weiter''. \\ \hline	
	Hauptszenario			& 1. Das System zeigt Verständnisfrage-Vorbereiten-Oberfläche. \\
							& 2. Der Benutzer tippt neue Frage, die Auswahlen für die 
							  Frage, sowie der Zeitstempel und bestätigt. \\ 
							& 3. Das System fügt die neue Frage in der Liste hinzu. \\ 
							& 4. Der Benutzer klickt auf ``Weiter''-Button. \\ 
							& 5. Das System zeigt TAN-Oberfläche an. \\ \hline
	Erweiterung				& 2a. Wenn der Benutzer noch nicht fertig mit der 
							  Verständnisfragen-Vorbereitung, geht er zurück zu Schritt 2. \\
							& 2b. Wenn die Frage leer ist oder es keine Auswahl für die 
							  Frage gibt oder der Zeitstempel nicht gültig ist, wird eine 
							  Fehlermeldung angezeigt. \\ \hline
	Priorität				& mittel \\ \hline
	Verwendungshäufigkeit	& normal \\ \hline
\end{tabularx}
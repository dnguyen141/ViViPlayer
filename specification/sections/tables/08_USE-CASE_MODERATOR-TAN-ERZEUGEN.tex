\begin{tabularx}{\linewidth}{|l|X|}
	\hline
	Use Case Nr. 08			& \textbf{Moderator TAN für ViViPlayer-Session Erzeugen} \\ \hline
	Erläuterungen			& TAN erzeugen ist der letzte Schritt zur Vorbereitung einer 
							  ViViPlayer-Session. \\ \hline
	Systemgrenzen (Scope)	& Applikation. \\ \hline
	Ebene					& Hauptebene. \\ \hline
	Vorbedingung			& Die Web-App ist betriebsbereit. Der Benutzer hat Moderator-
							  Rechten. \\ \hline
	Mindestgarantie			& Im Fehlerfall wird keine TAN sowie keine Session erstellt.
							  \\ \hline
	Erfolgsgarantie			& Eine sichere TAN wird für die Session erstellt. \\ \hline
	Stakeholder				& Der Benutzer - will eine produktive Session mit den Teilnehmer 
							  via ViViPlayer. \\
							& Herr Jianwei Shi - möchte alle Sessions in der Web-App 
							  erfolgreich durchgeführt werden. \\ \hline
	Hauptakteur				& Der Benutzer. \\ \hline
	Auslöser				& Der Benutzer hat schon ein ViVi ausgewählt, segmentiert sowie 
							  eine Liste von Verständnisfragen hinzugefügt, und klickt auf 
							  ``Weiter''. \\ \hline	
	Hauptszenario			& 1. Das System zeigt TAN-Oberfläche. \\
							& 2. Der Benutzer erzeugt eine neue TAN für die Session, und klickt auf das 
							  ``Weiter''-Button. \\ 
							& 3. Das System zeigt ViVi-Überblick-Oberfläche an. \\ \hline
	Erweiterung				& 2a. Alternativ kann man die automatische TAN-Erzeugung nutzen. 
							  \\
							& 2b. Wenn die TAN nicht sicher genug ist, wird eine Fehlermeldung ausgegeben. 
							  \\ \hline
	Priorität				& hoch \\ \hline
	Verwendungshäufigkeit	& häufig \\ \hline
\end{tabularx}
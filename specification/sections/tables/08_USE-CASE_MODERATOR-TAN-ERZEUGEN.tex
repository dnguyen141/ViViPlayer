\begin{tabularx}{\linewidth}{|l|X|}
	\hline
	Use Case Nr. 08			& \textbf{Moderator TAN für ViViPlayer-Session Erzeugen} \\ \hline
	Erläuterungen			& TAN erzeugen ist der letzte Schritt zur Vorbereitung einer 
							  ViViPlayer-Session. Danach kann die Teilnehmer mit TAN einloggen. \\ \hline
	Systemgrenzen (Scope)	& Gesamtsystem. \\ \hline
	Ebene					& Hauptfunktion. \\ \hline
	Vorbedingung			& Die Web-App ist betriebsbereit. Der Benutzer hat Moderator-
							  Rechte und schon ein ViVi ausgewählt (UC 3.2.5), segmentiert (UC 3.2.6) sowie eine Liste von Fragen hinzugefügt (UC 3.2.7), und klickt auf ``Weiter''. \\ \hline
	Mindestgarantie			& Im Fehlerfall wird keine TAN sowie keine Session erstellt.
							  \\ \hline
	Erfolgsfall 			& Eine sichere TAN ist für die Session erstellt. \\ \hline
	Stakeholder				& Systembediener - will eine produktive Session mit den Teilnehmer 
							  via ViViPlayer. \\
							& Herr Jianwei Shi (Systembesitzer) - möchte alle Sessions in der 
							  Web-App erfolgreich durchgeführt werden. \\ \hline
	Hauptakteur				& Systembediener mit Moderator-Rolle. \\ \hline
	Auslöser				& Der Benutzer befindet sich auf der ViVi-Fragen-Vorbereiten-Seite 
	                          und klickt auf ``Weiter''. \\ \hline	
	Hauptszenario			& 1. Der Benutzer befindet sich auf ViVi-Fragen-Vorbereiten-Seite
							  und klickt auf ``Weiter''. \\
							& 2. Das System zeigt TAN-Oberfläche. \\
							& 3. Der Benutzer erzeugt eine neue TAN für die Session, und klickt 
							  auf das ``Weiter''-Button. \\ 
							& 4. Das System zeigt ViVi-Überblick-Oberfläche an. \\ \hline
	Erweiterungen			& 3a. Alternativ kann man die automatische TAN-Erzeugung nutzen. 
							  \\
							& 3b. Wenn die TAN nicht sicher genug ist, wird eine Fehlermeldung 
							  ausgegeben. Das System geht zurück zu Schritt 2. \\ \hline
	Priorität				& hoch \\ \hline
	Verwendungshäufigkeit	& häufig \\ \hline
\end{tabularx}
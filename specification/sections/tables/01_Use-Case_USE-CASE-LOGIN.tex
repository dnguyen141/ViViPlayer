\begin{tabularx}{\linewidth}{|l|X|}
	\hline
	Use Case Nr. 01			& \textbf{ViViPlayer Log-in} \\ \hline
	Erläuterungen			& Damit der Benutzer alle Funktionen der App benutzen kann, 
							  muss er seine Identifikation bestätigen. \\ \hline
	Status					& Der Benutzer befindet sich in der Hauptseite der ViViPlayer-Web-App. \\ \hline
	Systemgrenzen (Scope)	& Login-System \\ \hline
	Ebene					& Hauptebene \\ \hline
	Vorbedingung			& Die Web-App ist betriebsbereit \\ \hline
	Mindestgarantie			& Das Log-in des Benutzers wird abgesagt, wenn seine Identität nicht im System ist. \\ \hline
	Erfolgsgarantie			& Die Identität des Benutzers wird erfolgreich bestätigt. Der Benutzer
							  wird zur Teilnehmer/Moderator-Hauptseiten weitergeleitet. \\ \hline
	Stakeholder				& Der Benutzer - möchte die Funktionen der Web-App schnell wie möglich nutzen. \newline
							  Jianwei Shi - möchte die Funktionen der Web-App für die Benutzer mit korrekter Identität
							  verfügbar sein.\\ \hline
	Hauptakteur				& Der Benutzer \\ \hline
	Auslöser				& Der Benutzer möchte einloggen \\ \hline	
	Hauptszenario			& 1. Das System fordert den Benutzer auf, seine Email-Adresse sowie Passwort einzugeben. \\
							& 2. Der Benutzer gibt seine Email-Adresse und Passwort ein. \\
							& 3. Das System validiert die vom Benutzer eingegebene Email-Adresse und Passwort und  stellt fest, welche Rolle der Benutzer hat (Teilnehmer oder Moderator) \\
							& 4. Den Benutzer wird im System angemeldet und weiter zur Teilnehmer/Moderator-Hauptseite
								 geleitet. \\ \hline
	Erweiterung				& 3a. Wenn das System feststellt, dass das Kennwort für den eingegebenen Benutzernamen falsch
							  ist, dann
							  \begin{itemize}
							  	\item wird den Benutzer auffordert, das Passwort erneut einzugeben 
							  	\item hat der Benutzer die Möglichkeit, ein vergessenes Passwort abzurufen 
							  \end{itemize} \\
							& 3b.  Wenn das System feststellt, dass der Benutzername für den eingegebenen Benutzernamen
							falsch ist, dann wird eine Fehlermeldung gezeigt. \\ \hline
	Priorität				& mittel \\ \hline
	Verwendungshäufigkeit	& hoch \\ \hline
\end{tabularx}

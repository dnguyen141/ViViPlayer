\begin{tabularx}{\linewidth}{|l|X|}
	\hline
	Use Case Nr. 08			& \textbf{ViVi steuern} \\ \hline
	Erläuterungen			& Als Moderator kann der Benutzer das ViVi steuern. \\ \hline
	Systemgrenzen (Scope)	& Gesamtsystem. \\ \hline
	Ebene					& Hauptfunktion. \\ \hline
	Vorbedingung			& Die Web-App ist betriebsbereit. Der Benutzer hat Moderator-
							  Rechte und hat schon eine ViViPlayer-Session erstellt (UC 5, UC 6, UC 7). Der ist in einer Session mit anderen Benutzern, die entweder Moderator oder Teilnehmer sind. \\ \hline
	Mindestgarantie			& Im Fehlerfall wird das ViVi beim Moderator oder bei Teilnehmer 
							  nicht navigiert. Das ViVi bleibt unverändert. \\ \hline
	Erfolgsfall 			& Das ViVi wurde für alle Benutzer in der Session navigiert. 
							  \\ \hline
	Stakeholder				& Systembediener (Moderator) - will eine produktive Session mit den 
							  Teilnehmer via ViViPlayer. \\
							& Systembesitzer (Systemadministrator) - möchte alle Sessions in der 
							  Web-App erfolgreich durchgeführt werden. \\ \hline
	Hauptakteur				& Systembediener (Moderator). \\ \hline
	Auslöser				& Der Benutzer navigiert das Video zu seinem gewünschten 
							  Zeitpunkt. \\ \hline	
	Hauptszenario			& 1. Der Benutzer navigiert das Video zu seinem gewünschten
							  Zeitpunkt. \\
							& 2. Das System erkennt die Eingabe des Benutzers und
							  das ViVi springt für alle Benutzer in der Session zu diesem Zeitpunkt.
							  \\ \hline
	Erweiterungen			&  \\ \hline
	Priorität				& sehr hoch \\ \hline
	Verwendungshäufigkeit	& sehr häufig \\ \hline
\end{tabularx}
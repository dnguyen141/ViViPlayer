\begin{tabularx}{\linewidth}{|l|X|}
	\hline
	Use Case Nr. 07			& \textbf{Moderator Fragen Erstellen} \\ \hline
	Erläuterungen			& Fragen erstellen ist der dritte Schritt zur Vorbereitung einer 
							  ViViPlayer-Session. \\ \hline
	Systemgrenzen (Scope)	& Gesamtsystem. \\ \hline
	Ebene					& Hauptfunktion. \\ \hline
	Vorbedingung			& Die Web-App ist betriebsbereit. Der Benutzer hat Moderator
							  -Rechte und schon ein ViVi ausgewählt (UC 3.2.5) und segmentiert (UC 3.2.6). \\ \hline
	Mindestgarantie			& Im Fehlerfall wird die Fragen nicht erstellt. 
							  Das ViVi wird nicht gelöscht. \\ \hline
	Erfolgsfall    			& Eine Liste von Verständnisfragen ist erfolgreich hinzugefürgt.
							  \\ \hline
	Stakeholder				& Systembediener - will eine produktive Session mit anderen 
	                          Benutzer via ViViPlayer. \\
							& Herr Jianwei Shi (Systembesitzer) - möchte, dass alle Sessions 
							  in der Web-App erfolgreich durchgeführt werden. \\ \hline
	Hauptakteur				& Systembediener mit Moderator-Rolle. \\ \hline
	Auslöser				& Der Benutzer befindet sich auf der ViVi-Segmentieren-Seite und 
							  klickt auf ``Weiter''. \\ \hline	
	Hauptszenario			& 1. Der Benutzer befindet sich auf der ViVi-Segmentieren-Seite 
	                          und klickt auf ``Weiter''. \\  
							& 2. Das System zeigt die Frage-Vorbereiten-Oberfläche. \\
							& 3. Der Benutzer tippt neue Frage, die Auswahlen für die 
							  Frage, sowie den Zeitstempel und den Typ der Frage ein und 
							  bestätigt. \\ 
							& 4. Das System fügt die neue Frage in der Liste hinzu. \\ 
							& 5. Der Benutzer klickt auf dem ``Weiter''-Button. \\ 
							& 6. Das System zeigt die TAN-Oberfläche an. \\ \hline
	Erweiterungen			& 3a. Wenn der Benutzer noch nicht fertig mit der 
							  Fragen-Vorbereitung ist, geht er zurück zu Schritt 3. \\
							& 3b. Wenn die Frage leer ist oder es keine Auswahl für die 
							  Frage gibt oder der Zeitstempel nicht gültig ist, wird eine 
							  Fehlermeldung angezeigt. Das System geht zurück zu Schritt 3.
							  \\ \hline
	Priorität				& mittel \\ \hline
	Verwendungshäufigkeit	& normal \\ \hline
\end{tabularx}
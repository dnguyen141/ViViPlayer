\begin{tabularx}{\linewidth}{|l|X|}
	\hline
	Use Case Nr. 07			& \textbf{Umfragen und Verstänisfragen erstellen} \\ \hline
	Erläuterungen			& Umfragen und Verstänisfragen erstellen ist der dritte Schritt zur
							  Vorbereitung einer ViViPlayer-Session. \\ \hline
	Systemgrenzen (Scope)	& Gesamtsystem. \\ \hline
	Ebene					& Hauptfunktion. \\ \hline
	Vorbedingung			& Die Web-App ist betriebsbereit. Der Benutzer hat Moderator
							  -Rechte und schon ein ViVi ausgewählt (UC 5) und segmentiert (UC 6). Er befindet sich auf der ViVi-Segmentieren-Seite. \\ \hline
	Mindestgarantie			& Im Fehlerfall wird die Um-/Verstänisfragen nicht erstellt. Das ViVi 
							  bleibt unverändert. \\ \hline
	Erfolgsfall    			& Eine Liste von Verständnisfragen ist erfolgreich hinzugefürgt.
							  \\ \hline
	Stakeholder				& Systembediener (Moderator) - will eine produktive Session mit 
							  anderen Benutzern via ViViPlayer. \\
							& Systembesitzer (Systemadministrator) - möchte, dass alle Sessions 
							  in der Web-App erfolgreich durchgeführt werden können. \\ \hline
	Hauptakteur				& Systembediener (Moderator). \\ \hline
	Auslöser				& Der Benutzer klickt auf den ``Weiter''-Button. \\ \hline	
	Hauptszenario			& 1. Der Benutzer klickt auf den ``Weiter''-Button. \\  
							& 2. Das System zeigt die Frage-Vorbereiten-Oberfläche. \\
							& 3. Der Benutzer tippt neue Frage, die Auswahlen für die 
							  Frage, sowie den Zeitstempel und den Typ der Frage ein und 
							  bestätigt. \\ 
							& 4. Das System fügt die neue Verstänis-/Umfrage in der Liste
							  hinzu. \\ 
							& 5. Der Benutzer klickt auf den ``Weiter''-Button. \\ 
							& 6. Das System zeigt die TAN-Oberfläche an. \\ \hline
	Erweiterungen			& 4a. WENN die Frage leer ist oder es keine Auswahl für die 
							  Frage gibt oder der Zeitstempel nicht gültig ist, DANN wird eine 
							  Fehlermeldung angezeigt. Zurück zu Schritt 2. \\ 
							& 5a. WENN der Benutzer noch nicht fertig mit der 
							  Fragen-Vorbereitung ist, DANN geht er zurück zu Schritt 3. \\ \hline
	Priorität				& mittel \\ \hline
	Verwendungshäufigkeit	& normal \\ \hline
\end{tabularx}
\begin{tabularx}{\linewidth}{|l|X|}
	\hline
	Use Case Nr. 13			& \textbf{An Um-/Verständnisfragen teilnehmen} \\ \hline
	Erläuterungen			& Während der Session haben alle Teilnehmr die Möglichkeit, an einer 
							  Um-/Verständnisfrage des Moderators teilzunehmen. \\ \hline
	Systemgrenzen (Scope)	& Gesamtsystem. \\ \hline
	Ebene					& Hauptfunktion. \\ \hline
	Vorbedingung			& Das System ist betriebsbereit. Der Benutzer befindet sich auf der 
							  ViViPlayer-Seite der Web-App. Da stellt der Moderator eine Um-/Verständnisfrage für alle Teilnehmer und das System zeigt die auf das Fenster für Um-/Verständnisfrage. \\ \hline
	Mindestgarantie			& Im Fehlerfall wird die Antwort des Benutzers nicht gezählt und 
							  nicht sichtbar zu den anderen sein. \\ \hline
	Erfolgsfall				& Der Moderator kann die Antwort vom Benutzer sehen oder die Antwort 
							  wird für das Balkendiagramm gerechnet. \\ \hline
	Stakeholder				& Systembediener (Teilnehmer) - möchte seine Meinung durch die 
							  Antwort für Um-/Verständnisfragen mitteilen. \\ 
							& Systembediener (Moderator) - möchte die Meinungen von Teilnehmer 
							  sammeln, damit die Lösung immer optimal ist. \\ \hline
	Hauptakteur				& Systembediener (Teilnehmer) \\ \hline
	Auslöser				& Der Benutzer klickt auf die Antwort der Um-/Verständnisfrage. \\ 
							  \hline	
	Hauptszenario			& 1. Der Benutzer klickt auf eine Antwort der Um-/Verständnisfrage, 
	                          2. Der Benutzer klickt auf den "Senden"-Button. \\
							& 3. Das System sammelt die Antwort von allen Benutzer. \\
							& 4. Das System zeigt nach 30 Sekunden es das Um-/Verständnisfragen im 
							  Form eines Balkendiagramm. \\ \hline
	Erweiterungen			& 3a. WENN der Benutzer keine Antwort eingibt, DANN zählt das System 
							  seine leere Antwort auch nicht. \\ \hline
	Priorität				& hoch \\ \hline
	Verwendungshäufigkeit	& regelmäßig \\ \hline
\end{tabularx}


\begin{tabularx}{\linewidth}{|l|X|}
	\hline
	Use Case Nr. 04			& \textbf{Moderator Session Vorbereiten} \\ \hline
	Erläuterungen			& Bevor ein Moderator eine Session mit den Teilnehmer anfangen 
	                          kann, muss er zuerst vorbereiten. Das heißt: er muss das ViVi einbetten und segmentieren, Verständnisfragen vorbereiten und TAN für die Session erzeugen. \\ \hline
	Systemgrenzen (Scope)	& Gesamtsystem. \\ \hline
	Ebene					& Hauptfunktion. \\ \hline
	Vorbedingung			& Die Web-App ist betriebsbereit. Der Benutzer hat 
	                          Moderator-Rechte. \\ \hline
	Mindestgarantie			& Im Fehlerfall wird keine Session erstellt.\\ \hline
	Erfolgsfall  			& Das segmentierte Vision-Video sowie eine sichere TAN sind 
	                          bereitgestellt für eine ViViPlayer-Session. \\ \hline
	Stakeholder				& Systembediener - will eine produktive Session mit den 
	                          Teilnehmer via ViViPlayer. \\
							& Herr Jianwei Shi (Systembesitzer) - möchte, dass alle Sessions 
							  in der Web-App erfolgreich durchgeführt werden. \\ \hline
	Hauptakteur				& Systembediener mit Moderator-Rolle. \\ \hline
	Auslöser				& Der Benutzer wählt die Möglichkeit, eine neue Session zu 
	                          erstellen, von der Moderator-Hauptseite. \\ \hline	
	Hauptszenario			& 1. Der Benutzer wählt die Möglichkeit, eine neue Session zu 
	                          erstellen, von der Moderator-Hauptseite. \\ 
							& 2. Das System zeigt die ViVi-Auswählen-Oberfläche an. \\
							& 3. Der Benutzer wählt ein ViVi und bestätigt. (UC 3.2.5) \\ 
							& 4. Das System zeigt die ViVi-Segmentieren-Oberfläche. \\ 
							& 5. Der Benutzer segmentiert das von ihm ausgewählte Video und 
							  bestätigt. (UC 3.2.6) \\
							& 6. Das System zeigt die Frage-Vorbereiten-Oberfläche. \\
							& 7. Der Benutzer schreibt die Multiple-Choice-Fragen für die 
							  Teilnehmer sowie den Typ der Frage und bestätigt. (UC 3.2.7) \\
							& 8. Das System zeigt die TAN-Oberfläche. \\
							& 9. Der Benutzer erzeugt eine neue TAN für die Session und 
							  bestätigt. (UC 3.2.8) \\ 
							& 10. Das System zeigt die ViVi-Überblick-Oberfläche. \\ 
							& 11. Der Benutzer prüft noch mal alle Vorbereitungschritte und 
							  erstellt neue Session. \\
							& 12. Das System zeigt die ViViPlayer-Oberfläche. \\ \hline
	Erweiterungen			& 4a. Falls das Video nicht eingebettet werden kann, wird eine 
	                          Fehlermeldung gezeigt. Das System geht zurück zu Schritt 2. \\
							& 6a. Falls die Segmentierung nicht gültig ist, wird eine 
							  Fehlermeldung ausgegeben. Das System geht zurück zu Schritt 4.\\
							& 8a. Falls es Fehler bei der Eingabe der Fragen, wird 
							  eine Fehlermeldung ausgegeben. Das System geht zurück zu Schritt 4. \\
							& 10a. Wenn die TAN nicht sicher genug ist, wird eine 
							  Fehlermeldung ausgegeben. Das System geht zurück zu Schritt 8. 
							  \\ \hline
	Priorität				& hoch \\ \hline
	Verwendungshäufigkeit	& häufig \\ \hline
\end{tabularx}
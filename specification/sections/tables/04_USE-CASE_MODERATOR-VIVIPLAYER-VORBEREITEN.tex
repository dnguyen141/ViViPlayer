\begin{tabularx}{\linewidth}{|l|X|}
	\hline
	Use Case Nr. 04			& \textbf{Moderator Vision-Video Vorbereiten} \\ \hline
	Erläuterungen			& Bevor ein Moderator eine Session mit den Teilnehmer anfangen kann,
							  muss er zuerst vorbereiten. \\ \hline
	Systemgrenzen (Scope)	& Applikation. \\ \hline
	Ebene					& Hauptebene. \\ \hline
	Vorbedingung			& Die Web-App ist betriebsbereit. Der Benutzer hat Moderator-Rechten. \\ \hline
	Mindestgarantie			& Im Fehlerfall wird keine Session erstellt.\\ \hline
	Erfolgsgarantie			& Das segmentierte Vision-Video sowie eine sichere TAN sind bereitgestellt für
							  eine ViViPlayer-Session. \\ \hline
	Stakeholder				& Der Benutzer - will eine produktive Session mit den Teilnehmer via
							  ViViPlayer. \\
							& Herr Jianwei Shi - möchte alle Sessions in der Web-App erfolgreich durchgeführt
							  werden. \\ \hline
	Hauptakteur				& Der Benutzer. \\ \hline
	Auslöser				& Der Benutzer wählt die Möglichkeit, eine neue Session zu erstellen, von der
							  Moderator-Hauptseite. \\ \hline	
	Hauptszenario			& 1. Das System zeigt ViVi-Auswählen-Oberfläche an. \\
							& 2. Der Benutzer wählt ein ViVi und bestätigt. \\ 
							& 3. Das System zeigt ViVi-Segmentieren-Oberfläche. \\ 
							& 4. Der Benutzer segmentiert das von ihm ausgewählte Video und bestätigt. \\
							& 5. Das Systen zeigt Verständnisfrage-Vorbereiten-Oberfläche. \\
							& 6. Der Benutzer schreibt die Multiple-Choice-Fragen für die Teilnehmer und 
							  bestätigt. \\
							& 7. Das System zeigt TAN-Oberfläche an. \\
							& 8. Der Benutzer erzeugt eine neue TAN für die Session und bestätigt. \\ 
							& 9. Das System zeigt ViVi-Überblick-Oberfläche an. \\ 
							& 10. Der Benutzer prüft noch mal die Einstellung und erstellt neue Session. \\
							& 11. Das System zeigt ViViPlayer-Oberfläche an. \\ \hline
	Erweiterung				& 3a. Falls das Video nicht eingebettet werden kann, wird eine Fehlermeldung 
							  gezeigt. \\
							& 5a. Falls die Segmentierung nicht gültig ist, wird eine Fehlermeldung 
							  ausgegeben. \\
							& 11a. Wenn die TAN nicht sicher genug ist, wird eine Fehlermeldung ausgegeben.
							  \\ \hline
	Priorität				& hoch \\ \hline
	Verwendungshäufigkeit	& häufig \\ \hline
\end{tabularx}
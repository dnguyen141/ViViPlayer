\begin{tabularx}{\linewidth}{|l|X|}
	\hline
	Use Case Nr. 04			& \textbf{Session vorbereiten} \\ \hline
	Erläuterungen			& Bevor ein Moderator eine Session mit den Teilnehmern anfangen
	                          kann, muss er zuerst sie vorbereiten. Das heißt: er muss das ViVi einbetten und segmentieren, Verständnisfragen vorbereiten und TAN für die Session erzeugen, damit er die TAN für die Teilnehmer der Session schicken kann. \\ \hline
	Systemgrenzen (Scope)	& Gesamtsystem. \\ \hline
	Ebene					& Hauptfunktion. \\ \hline
	Vorbedingung			& Die Web-App ist betriebsbereit. Der Benutzer hat 
	                          Moderator-Rechte. Er befindet sich auf der Moderator-Hauptseite. \\ \hline
	Mindestgarantie			& Im Fehlerfall wird keine Session erstellt.\\ \hline
	Erfolgsfall  			& Das segmentierte Vision-Video sowie eine sichere TAN sind 
	                          bereitgestellt für eine ViViPlayer-Session. \\ \hline
	Stakeholder				& Systembediener (Moderator) - will eine produktive Session mit den 
	                          Teilnehmern via ViViPlayer. \\
							& Systembesitzer (Systemadministrator) - möchte, dass alle Sessions 
							  in der Web-App erfolgreich durchgeführt werden. \\ \hline
	Hauptakteur				& Systembediener (Moderator). \\ \hline
	Auslöser				& Der Benutzer klickt auf die ``Session erstellen''-Option. \\ \hline	
	Hauptszenario			& 1. Der Benutzer klickt auf die ``Session erstellen''-Option. \\ 
							& 2. Das System zeigt die ViVi-Auswählen-Oberfläche an. \\
							& 3. Der Benutzer wählt ein ViVi und bestätigt. (UC 5) \\ 
							& 4. Das System zeigt die ViVi-Segmentieren-Oberfläche. \\ 
							& 5. Der Benutzer segmentiert das von ihm ausgewählte Video und 
							  bestätigt. (UC 6) \\
							& 6. Das System zeigt die Frage-Vorbereiten-Oberfläche. \\
							& 7. Der Benutzer bereitet die Umfrage/Verstänisfragen vor und bestätigt.
							  (UC 7) \\
							& 8. Das System zeigt die ViVi-Überblick-Oberfläche. \\ 
							& 9. Der Benutzer prüft noch mal alle Vorbereitungschritte und 
							  klickt auf den ``Erstellen''-Button. \\
							& 10. Das System erstellt eine neue Session und eine sichere TAN. \\ 
							  \hline
	Erweiterungen			& 4a. WENN das Video nicht eingebettet werden kann, DANN wird eine 
	                          Fehlermeldung gezeigt. Zurück zu Schritt 2. \\
							& 6a. WENN die Segmentierung nicht gültig ist, DANN wird eine 
							  Fehlermeldung ausgegeben. Zurück zu Schritt 4.\\
							& 8a. WENN es Fehler bei der Eingabe der Fragen, DANN wird 
							  eine Fehlermeldung ausgegeben. Zurück zu Schritt 4. \\ \hline
	Priorität				& sehr hoch \\ \hline
	Verwendungshäufigkeit	& häufig \\ \hline
\end{tabularx}
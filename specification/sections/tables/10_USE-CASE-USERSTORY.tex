\begin{tabularx}{\linewidth}{|l|X|}
	\hline
	Use Case Nr. 10			& \textbf{User-Story schreiben} \\ \hline
	Erläuterungen			& Systembediener möchte eine User Story schreiben. \\ \hline
	Systemgrenzen (Scope)	& User-Story-System. \\ \hline
	Ebene					& Hauptfunktion. \\ \hline
	Vorbedingung			& Das System ist betriebsbereit. Der Benutzer hat Moderator- oder 
							  Teilnehmer-Rolle, der befindet sich in einer ViViPlayer-Session. \\ \hline
	Mindestgarantie			& Im Fehlerfall erhält der Benutzer eine Fehlermeldung und es 
							  werden keine Daten in dem Server abgespeichert. \\ \hline
	Erfolgsgarantie			& Der Benutzer hat eine User Story verfasst und der Server hat
							  diese abgespeichert. Die User Story wurde einem Shot
							  zugeordnet. \\ \hline
	Stakeholder				& Systembediener (Moderator/Teilnehmer) - möchte User Stories 
							  einfach verfassen können.\\ 
                            & Entwickler - möchten User Stories für ihr Projekt haben. \\ 
                              \hline
	Hauptakteur				& Systembediener (Moderator/Teilnehmer). \\ \hline
	Auslöser				& Der Benutzer klickt auf das Textfeld für User Story und gibt 
							  seine User Story ein, dann bestätigt. \\ \hline	
	Hauptszenario			& 1. Der Benutzer klickt auf das Textfeld für User Story und gibt 
							  seine User Story ein, dann bestätigt. \\
							& 2. Das System verarbeitet diese Anfrage und speichert die User 
							  Story ab. Zudem wird sie dem jeweiligen Shot zugeordnet. \\
							& 3. Das System zeigt die neue User Story den anderen Nutzern an.\\
							& 4. Der Nutzer kann eine weitere User Story verfassen. \\ \hline
	Erweiterungen			& 2a. WENN ein Fehler auftritt DANN wird der Benutzer 
	                          benachrichtigt und er kann zurück zu Schritt 1 gehen. Es werden keine Daten in der Datenbank abgespeichert. \\ \hline
	Priorität				& hoch \\ \hline
	Verwendungshäufigkeit	& häufig \\ \hline
\end{tabularx}

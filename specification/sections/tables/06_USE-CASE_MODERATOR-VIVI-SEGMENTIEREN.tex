\begin{tabularx}{\linewidth}{|l|X|}
	\hline
	Use Case Nr. 06			& \textbf{Moderator Vision-Video Segmentieren} \\ \hline
	Erläuterungen			& Vision-Video segmentieren ist der zweite Schritt zur 
							  Vorbereitung einer ViViPlayer-Session. \\ \hline
	Systemgrenzen (Scope)	& Gesamtsystem. \\ \hline
	Ebene					& Hauptfunktion \\ \hline
	Vorbedingung			& Die Web-App ist betriebsbereit. Der Benutzer hat 
	                          Moderator-Rechte und schon ein ViVi ausgewählt (UC 3.2.5). \\ \hline
	Mindestgarantie			& Im Fehlerfall wird das Video nicht segmentiert und nicht 
	                          gelöscht. \\ \hline
	Erfolgsfall 			& Das ViVi wurde erfolgreich segmentiert. \\ \hline
	Stakeholder				& Systembediener - will eine produktive Session mit den 
	                          Teilnehmer via ViViPlayer. \\
							& Herr Jianwei Shi (Systembesitzer)- möchte, dass alle Sessions 
							  in der Web-App erfolgreich durchgeführt werden. \\ \hline
	Hauptakteur				& Systembediener mit Moderator-Rolle \\ \hline
	Auslöser				& Der Benutzer befindet sich auf der ViVi-Auswählen-Seite 
							  und klickt auf ``Weiter''-Button. \\ \hline	
	Hauptszenario			& 1. Der Benutzer befindet sich auf der ViVi-Auswählen-Seite
							  und klickt auf ``Weiter''-Button. \\
							& 2. Das System zeigt die ViVi-Segmentieren-Oberfläche. \\
							& 3. Von dem Video-Fenster spielt der Benutzer das Video ab, bis 
							  zu seinem gewünschten Zeitstempel für ein neues Shot, danach 
							  benennt er das Shot nach seinem Wunsch und klickt 
							  ``Hinzufügen''. \\
							& 4. Das System fügt das neue Shot in der Shots-Liste 
							  hinzu. \\
							& 5. Der Benutzer klickt auf dem ``Weiter''-Button. \\
							& 6. Das System zeigt die Verständnisfrage-Vorbereiten-Oberfläche 
							  an. \\ \hline
	Erweiterungen			& 3a. Alternativ kann der Benutzer die automatische 
							  Segmentierung nutzen. \\
							& 4a. Wenn der Benutzer noch nicht fertig mit dem Segmentierung 
							  ist, geht der Benutzer zurück zu Schritt 2. \\ 
							& 4b. Falls der Name eines Shots leer oder eine Zeitstempel 
							  nicht gültig ist, wird eine Fehlermeldung ausgegeben. Kein neues Shot wird hinzugefügt. Das System geht zurück zu Schritt 2. \\ \hline
	Priorität				& sehr hoch \\ \hline
	Verwendungshäufigkeit	& häufig \\ \hline
\end{tabularx}
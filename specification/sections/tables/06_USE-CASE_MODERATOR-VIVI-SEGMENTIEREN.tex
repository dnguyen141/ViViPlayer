\begin{tabularx}{\linewidth}{|l|X|}
	\hline
	Use Case Nr. 06			& \textbf{ViVi segmentieren} \\ \hline
	Erläuterungen			& Vision-Video segmentieren ist der zweite Schritt zur 
							  Vorbereitung einer ViViPlayer-Session. \\ \hline
	Systemgrenzen (Scope)	& Gesamtsystem. \\ \hline
	Ebene					& Hauptfunktion \\ \hline
	Vorbedingung			& Die Web-App ist betriebsbereit. Der Benutzer hat 
	                          Moderator-Rechte und schon ein ViVi ausgewählt (UC 5). Er befindet sich auf der ViVi-Auswählen-Seite. \\ \hline
	Mindestgarantie			& Im Fehlerfall wird das Video nicht segmentiert. Das ausgewählte 
							  ViVi bleibt unverändert. \\ \hline
	Erfolgsfall 			& Das ViVi wurde erfolgreich segmentiert. \\ \hline
	Stakeholder				& Systembediener (Moderator) - will eine produktive Session mit den 
	                          Teilnehmer via ViViPlayer. \\
							& Systembesitzer (Herr Jianwei Shi) - möchte, dass alle Sessions 
							  in der Web-App erfolgreich durchgeführt werden können. \\ \hline
	Hauptakteur				& Systembediener (Moderator) \\ \hline
	Auslöser				& Der Benutzer klickt auf den ``Weiter''-Button. \\ \hline	
	Hauptszenario			& 1. Der Benutzer klickt auf den ``Weiter''-Button. \\
							& 2. Das System zeigt die ViVi-Segmentieren-Oberfläche an. \\
							& 3. Von dem Video-Fenster spielt der Benutzer das Video ab, bis 
							  zu seinem gewünschten Zeitstempel für ein neuen Shot, danach
							  benennt er den Shot nach Wunsch und klickt auf den ``Hinzufügen''-Button. \\
							& 4. Das System fügt den neuen Shot in der Shots-Liste
							  hinzu. \\
							& 5. Der Benutzer klickt auf den ``Weiter''-Button. \\
							& 6. Das System zeigt die Verständnisfrage-Vorbereiten-Oberfläche 
							  an. \\ \hline
	Erweiterungen			& 3a. Alternativ kann der Benutzer die automatische 
							  Segmentierung nutzen. \\
							& 4a. WENN der Name eines Shots leer oder ein Zeitstempel
							  nicht gültig ist, DANN wird eine Fehlermeldung ausgegeben. Kein neuer Shot wird hinzugefügt. Zurück zu Schritt 2. \\
							& 5a. WENN der Benutzer noch nicht fertig mit dem Segmentierung 
							  ist, DANN geht er zurück zu Schritt 3. \\ \hline
	Priorität				& sehr hoch \\ \hline
	Verwendungshäufigkeit	& häufig \\ \hline
\end{tabularx}
\begin{tabularx}{\linewidth}{|l|X|}
	\hline
	Use Case Nr. 14			& \textbf{Annotation erstellen} \\ \hline
	Erläuterungen			& Während der Session haben die Benutzer mit Moderator-Rolle die 
							  Möglichkeit, Annotation hinzuzufügen. \\ \hline
	Systemgrenzen (Scope)	& Gesamtsystem. \\ \hline
	Ebene					& Hauptfunktion. \\ \hline
	Vorbedingung			& Das System ist betriebsbereit. Der Benutzer befindet sich auf 
							  der ViViPlayer-Seite der Web-App mit anderen Benutzer. \\ \hline
	Mindestgarantie			& Im Fehlerfall wird keine Annotation erstellt und eine 
							  Fehlermeldung wird ausgegeben. \\ \hline
	Erfolgsfall				& Eine Annotation wird auf das Video-Fenster hinzugefügt. \\ 
							  \hline
	Stakeholder				& Systembediener (Moderator/Teilnehmer) - möchte zusätzliche 
							  Information über die Inhalt des ViVis hinzufügen. \\ \hline
	Hauptakteur				& Systembediener (Moderator) \\ \hline
	Auslöser				& Der Benutzer klickt auf die ``Annotation hinzufügen''-Button. 
							  \\ \hline	
	Hauptszenario			& 1. Der Benutzer klickt auf die ``Annotation hinzufügen''-Button. 
							  \\
							& 2. Das System fügt ein neues Annotation-Pop-up auf dem 
							  Video-Fenster hinzu. \\
							& 3. Der Benutzer schiebt das Pop-up zu der gewünschten Position. \\
							& 4. Der Benutzer tippt die Beschreibung ein und bestätigt. \\
							& 5. Das System fügt das Pop-up als neue Annotation hinzu. \\ \hline
	Erweiterungen			& 3a. Alternativ kann der Benutzer auf den ``Abbrechen''-Button klicken.\\ 
							  \hline
	Priorität				& niedrig \\ \hline
	Verwendungshäufigkeit	& weniger häufig  \\ \hline
\end{tabularx}


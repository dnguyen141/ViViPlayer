\begin{tabularx}{\linewidth}{|l|X|}
	\hline
	Use Case Nr. 02			& \textbf{ViViPlayer Sign-up} \\ \hline
	Erläuterungen			& Jeder Benutzer braucht eine Konto, um im System anzumelden und damit die Funktionen
							  der Web-App nutzen kann. \\ \hline
	Status					& Der Benutzer befindet sich in der Hauptseite der ViViPlayer-Web-App. \\ \hline
	Systemgrenzen (Scope)	& Register-System \\ \hline
	Ebene					& Hauptebene \\ \hline
	Vorbedingung			& Die Web-App ist betriebsbereit \\ \hline
	Mindestgarantie			& Die Registrierung wird abgesagt, wenn die eingegebenen Daten vom Benutzer nicht gültig
							  ist, keine Daten wird im Datenbank abgespeichert. \\ \hline
	Erfolgsgarantie			& Die Identität des Benutzers wird erfolgreich im System abgespeichert. Der Benutzer
							  wird zur Hauptseite weitergeleitet, damit er mit der neuen Konto einloggen kann. \\ \hline
	Stakeholder				& Der Benutzer - möchte die Funktionen der Web-App schnell wie möglich nutzen. \newline
							  Jianwei Shi - möchte die Funktionen der Web-App für die Benutzer mit korrekter Identität
							  verfügbar sein.\\ \hline
	Hauptakteur				& Der Benutzer \\ \hline
	Auslöser				& Der Benutzer möchte registrieren und klickt auf "Registration"-Button \\ \hline	
	Hauptszenario			& 1. Das System fordert den Benutzer auf, seine Daten einzugeben, damit es seine
							  Identität im System speichert kann. \\
							& 2. Der Benutzer gibt alle seine Daten ein: Email-Adresse, Vorname, Nachname, Passwort 
							  und Passwort-Wiederholung.\\
							& 3. Das System validiert die vom Benutzer Daten. Wenn alles klappt, wird die Daten
							  im Datenbank gespeichert, und ein Email wird dem Benutzer geschickt, damit er selbst
							  für die Registrierung bestätigen kann. \\
							& 4. Den Benutzer wird im System angemeldet und weiter zur Teilnehmer/Moderator-Hauptseite
							  geleitet. \\ \hline
	Erweiterung				& 3a. Wenn der Vorname des Benutzers nicht eingegeben wird, funktioniert die Registrierung
							  wie normal. \\
							& 3b. Wenn das System feststellt, dass es noch mindestens ein Stück von Information (nicht
							  Vorname) fehlt, dann wird eine Fehlermeldung ausgegeben. Der Benutzer muss alle fehlende Information sowie die Passwort und Passwort-Wiederholung noch einmal eingeben, wenn er noch registrieren möchte. \\ \hline
	Priorität				& mittel \\ \hline
	Verwendungshäufigkeit	& mittel \\ \hline
\end{tabularx}

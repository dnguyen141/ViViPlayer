\begin{tabularx}{\linewidth}{|l|X|}
	\hline
	Use Case Nr. 01			& \textbf{Moderator Registrieren} \\ \hline
	Erläuterungen			& Um alle Rechte eines Moderators zu bekommen, muss ein 
							Benutzer als Moderator registrieren. \\ \hline
	Systemgrenzen (Scope)	& Registrierung-System. \\ \hline
	Ebene					& Hauptebene \\ \hline
	Vorbedingung			& Die Web-App ist betriebsbereit. \\ \hline
	Mindestgarantie			& Keine Moderator-Konto wird im System registriert. \\ \hline
	Erfolgsgarantie			& Eine neue Moderator-Konto wird erstellt. \\ \hline
	Stakeholder				& Der Benutzer - möchte die Funktionen der Web-App schnell 
							  wie möglich nutzen. \\
							& Herr Jianwei Shi - möchte die Funktionen der Web-App für die 
							  Benutzer mit korrektem Zugriffrecht verfügbar sein.\\ \hline
	Hauptakteur				& Der Benutzer \\ \hline
	Auslöser				& Der Benutzer möchte als Moderator registrieren und klickt 
							  auf ``Registrieren''-Button in der Login-Seite. \\ \hline	
	Hauptszenario			& 1. Das System fordert den Benutzer auf, eine Email-Adresse 
							  sowie die Passwort für die neue Konto einzugeben. \\
							& 2. Der Benutzer gibt seine Email-Adresse und seine Passwort 
							  ein und bestätigt \\
							& 3. Das System und ein anderer Moderator validiert die vom 
							  Benutzer eingegebene Daten. Eine Meldung wird als 
							  Anleitung angezeigt, was der neue Benutzer weiter machen soll. \\
							& 4. Dem Benutzer wird zurück zur Hauptseite geleitet. 
							  \\ \hline
	Erweiterung				& 3a. Wenn das System feststellt, dass die eingegebene Daten 
							  (Benutzername/Email-Adresse und Passwort) nicht abgestimmt sind, wird eine Fehlermeldung ausgegeben. \\ \hline
	Priorität				& niedrig \\ \hline
	Verwendungshäufigkeit	& selten \\ \hline
\end{tabularx}

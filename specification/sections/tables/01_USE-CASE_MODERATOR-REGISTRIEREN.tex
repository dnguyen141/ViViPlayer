\begin{tabularx}{\linewidth}{|l|X|}
	\hline
	Use Case Nr. 01			& \textbf{Moderator Registrieren} \\ \hline
	Erläuterungen			& Um alle Rechte eines Moderators zu bekommen, muss ein 
							Benutzer als Moderator registriert werden. \\ \hline
	Systemgrenzen (Scope)	& Registrierung-System. \\ \hline
	Ebene					& Hauptfunktion \\ \hline
	Vorbedingung			& Die Web-App ist betriebsbereit. \\ \hline
	Mindestgarantie			& Keine neue Moderator-Konto wird im System registriert. 
							  Klare Fehlermeldung wird ausgegeben. \\ \hline
	Erfolgsfall  			& Ein neues Moderator-Konto wurde erstellt. \\ \hline
	Stakeholder				& Systembediener - möchte die Funktionen der Web-App so schnell 
							  wie möglich nutzen. \\
							& Herr Jianwei Shi (Systembesitzer) - möchte die Funktionen der 
							  Web-App für die Benutzer mit korrektem Zugriffsrecht verfügbar sein.\\ \hline
	Hauptakteur				& Systembediener, die als Moderator registriert werden möchten \\ 
	                          \hline
	Auslöser				& Der Benutzer möchte als Moderator für ViViPlayer-Web-App 
							  registriert werden und klickt auf ``Registrieren''-Button in der Hauptseite von ViViPlayer-Web-App. \\ \hline	
	Hauptszenario			& 1. Der Benutzer klickt auf ``Registrieren''-Button in der 
							  Hauptseite von ViViPlayer-Web-App. \\
							& 2. Das System fordert den Benutzer auf, eine Email-Adresse 
							  sowie die Passwort für die neue Konto einzugeben. \\
							& 3. Der Benutzer gibt seine Email-Adresse und seine Passwort 
							  ein und bestätigt \\
							& 4. Das System und ein anderer Moderator validiert die vom 
							  Benutzer eingegebene Daten. Eine Meldung wird als 
							  Anleitung angezeigt, was der neue Benutzer weiter machen soll. \\
							& 5. Der Benutzer wird zurück zur Hauptseite geleitet. 
							  \\ \hline
	Erweiterungen			& 4a. WENN das System feststellt, dass die eingegebene Daten 
							  (Benutzername/Email-Adresse und Passwort) nicht abgestimmt sind, DANN wird eine Fehlermeldung ausgegeben. Das System geht zurück zu Schritt 2. \\ \hline
	Priorität				& niedrig \\ \hline
	Verwendungshäufigkeit	& kaum \\ \hline
\end{tabularx}

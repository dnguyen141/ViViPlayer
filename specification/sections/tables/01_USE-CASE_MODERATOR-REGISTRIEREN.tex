\begin{tabularx}{\linewidth}{|l|X|}
	\hline
	Use Case Nr. 01			& \textbf{Moderator registrieren} \\ \hline
	Erläuterungen			& Um alle Rechte eines Moderators zu bekommen, muss ein 
							Benutzer als Moderator registriert werden. \\ \hline
	Systemgrenzen (Scope)	& Registrierung-System. \\ \hline
	Ebene					& Hauptfunktion \\ \hline
	Vorbedingung			& Die Web-App ist betriebsbereit. Der Benutzer befindet sich auf die 
							  Hauptseite von ViViPlayer-Web-App. \\ \hline
	Mindestgarantie			& Im Fehlerfall wird kein neues Moderator-Konto im System 
	                          registriert. Klare Fehlermeldung wird ausgegeben. \\ \hline
	Erfolgsfall  			& Ein neues Moderator-Konto wurde erstellt. \\ \hline
	Stakeholder				& Systembediener (Moderator) - möchte die Funktionen der Web-App so 
							  schnell wie möglich nutzen. \\
							& Systembesitzer (Systemadministrator) - möchte, dass die Funktionen 
							  der Web-App für die Benutzer mit korrekten Zugriffsrechten verfügbar sind.\\ \hline
	Hauptakteur				& Systembediener (Moderator) \\ \hline
	Auslöser				& Der Benutzer klickt auf den ``Registrieren''-Button. \\ \hline	
	Hauptszenario			& 1. Der Benutzer klickt auf den ``Registrieren''-Button. \\
							& 2. Das System fordert den Benutzer auf, eine Email-Adresse 
							  sowie die Passwort für das neue Konto einzugeben. \\
							& 3. Der Benutzer gibt seinen Benutzername und seine Passwort 
							  ein und bestätigt sie \\
							& 4. Das System und ein anderer Moderator validiert die vom 
							  Benutzer eingegebene Daten. Eine Meldung wird als 
							  Anleitung angezeigt, was der neue Benutzer weiter machen soll. \\
							& 5. Der Benutzer wird zurück zur Hauptseite geleitet. 
							  \\ \hline
	Erweiterungen			& 4a. WENN das System feststellt, dass die eingegebene Daten 
							  (Benutzername und Passwort) nicht gültig sind, DANN wird eine Fehlermeldung ausgegeben. Zurück zu Schritt 2. \\ \hline
	Priorität				& niedrig \\ \hline
	Verwendungshäufigkeit	& weniger häufig \\ \hline
\end{tabularx}

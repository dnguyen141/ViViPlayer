\begin{tabularx}{\linewidth}{|l|X|}
	\hline
	Use Case Nr. 05			& \textbf{Moderator Vision-Video Auswählen} \\ \hline
	Erläuterungen			& Vision-Video auswählen ist der erste Schritt zur Vorbereitung 
							  einer ViViPlayer-Session. \\ \hline
	Systemgrenzen (Scope)	& Applikation. \\ \hline
	Ebene					& Hauptebene \\ \hline
	Vorbedingung			& Die Web-App ist betriebsbereit. Der Benutzer hat 
							  Moderator-Rechten. \\ \hline
	Mindestgarantie			& Im Fehlerfall wird kein Video ausgewählt und somit keine 
							  Session.\\ \hline
	Erfolgsgarantie			& Der Benutzer kann sein gewünschtes Video problemlos auswählen. 
							  \\ \hline
	Stakeholder				& Der Benutzer - will eine produktive Session mit den Teilnehmer 
							  via ViViPlayer. \\
							& Herr Jianwei Shi - möchte alle Sessions in der Web-App 
							  erfolgreich durchgeführt werden. \\ \hline
	Hauptakteur				& Der Benutzer. \\ \hline
	Auslöser				& Der Benutzer wählt die Möglichkeit, eine neue Session zu 
							  erstellen, von der Moderator-Hauptseite. \\ \hline	
	Hauptszenario			& 1. Das System zeigt ViVi-Auswählen-Oberfläche an. \\
							& 2. Der Benutzer kann ein ViVi vom Cloud auswählen und klickt auf das ``Weiter''-Button, wenn er schon fertig ist. \\
							& 3. Das System zeigt ViVi-Segmentieren-Oberfläche an. \\ \hline
	Erweiterung				& 2a. Alternativ kann das Video via URL eingebettet werden. \\ 
							& 3a. Falls das Video nicht ausgewählt werden kann, wird eine 
							  Fehlermeldung gezeigt. \\ \hline
	Priorität				& sehr hoch \\ \hline
	Verwendungshäufigkeit	& häufig \\ \hline
\end{tabularx}
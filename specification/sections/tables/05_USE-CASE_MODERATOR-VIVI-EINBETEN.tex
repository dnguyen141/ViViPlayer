\begin{tabularx}{\linewidth}{|l|X|}
	\hline
	Use Case Nr. 05			& \textbf{ViVi auswählen} \\ \hline
	Erläuterungen			& Vision-Video auswählen ist der erste Schritt zur Vorbereitung 
							  einer ViViPlayer-Session. \\ \hline
	Systemgrenzen (Scope)	& Gesamtsystem, lokaler File-Explorer. \\ \hline
	Ebene					& Hauptebene \\ \hline
	Vorbedingung			& Die Web-App ist betriebsbereit. Der Benutzer hat 
							  Moderator-Rechte und die Option gewählt, eine neue Session zu 
							  erstellen, auf der Moderator-Hauptseite. \\ \hline
	Mindestgarantie			& Im Fehlerfall wird kein Video ausgewählt und somit keine 
							  Session.\\ \hline
	Erfolgsfall 			& Der Benutzer konnte sein gewünschtes Video problemlos auswählen. 
							  \\ \hline
	Stakeholder				& Systembediener (Moderator) - will eine produktive Session mit den 
							  Teilnehmer via ViViPlayer. \\
							& Systembesitzer (Herr Jianwei Shi) - möchte alle Sessions in der 
							  Web-App erfolgreich durchgeführt werden. \\ \hline
	Hauptakteur				& Systembediener (Moderator). \\ \hline
	Auslöser				& Der Benutzer klickt auf die ``Session erstellen''-Option. \\ \hline	
	Hauptszenario			& 1. Der Benutzer klickt auf die ``Session erstellen''-Option. \\
							& 2. Das System zeigt die ViVi-Auswählen-Oberfläche an. \\
							& 3. Der Benutzer klickt auf das ``Hochladen''-Option. \\
							& 4. Das System zeigt ein Fenster zum Hochladen des Vivis an. \\
							& 5. Der Benutzer wählt ein ViVi von seinem Dateisystem aus und 
							  bestätigt. \\
							& 6. Das System zeigt die ViVi-Segmentieren-Oberfläche an. \\ \hline
	Erweiterungen			& 3a. Alternativ kann der Benutzer auf den ``Abbrechen''-Button klicken. \\
							& 4a. WHEN der Benutzer auf den ``Abbrechen''-Button klickt, DANN wird das 
							  Hochladen-Fenster abgeschlossen. Zurück zu Schritt 2  \\
							& 6a. WENN das Video nicht ausgewählt werden kann, DANN wird eine 
							  Fehlermeldung gezeigt. Zurück zu Schritt 4. \\ \hline
	Priorität				& sehr hoch \\ \hline
	Verwendungshäufigkeit	& häufig \\ \hline
\end{tabularx}
\begin{tabularx}{\linewidth}{|l|X|}
	\hline
	Use Case Nr. 11			& \textbf{In Session kommentieren} \\ \hline
	Erläuterungen			& Der Benutzer möchte einen Kommentar/Satz verfassen. \\ \hline
	Systemgrenzen (Scope)	& Kommentar-System. \\ \hline
	Ebene					& Hauptfunktion. \\ \hline
	Vorbedingung			& Das System ist betriebsbereit. Der Benutzer hat Moderator- oder 
							  Teilnehmer-Rechte, er befindet sich auf der ViViPlayer-Seite der Web-App. \\ \hline
	Mindestgarantie			& Der Benutzer erhält eine Fehlermeldung und es werden keine Daten 
							  in der Datenbank abgespeichert. Die Daten von anderen Benutzern werden abgespeichert. \\ \hline
	Erfolgsfall				& Der Benutzer hat eine Nachricht verfasst und der Server hat 
							  diese abgespeichert. Die Nachricht wurde einem Segment zugeordnet.\\ \hline
	Stakeholder				& Systembediener (Moderator/Teilnehmer) - möchte Kommentare einfach 
							  verfassen oder Feedback geben können.\\ 
                            & Entwickler - möchten Kommentare für ihr Projekt haben und 
                              verfassen können. \\ \hline
	Hauptakteur				& Systembediener (Moderator/Teilnehmer) \\ \hline
	Auslöser				& Der Benutzer wählt den Kommentar-Modus aus. \\ \hline	
	Hauptszenario			& 1. Der Benutzer wählt den Kommentar-Modus aus. \\
                            & 2. Der Benutzer gibt seinen Kommentar ein und bestätigt, wenn er 
                              fertig ist\\
							& 3. Das System verarbeitet diese Anfrage und speichert den 
							  Kommentar ab. Zudem wird er dem jeweiligen Segment zugeordnet. \\
							& 4. Das System zeigt den neuen Kommentar den anderen Nutzern an. \\ \hline
	Erweiterungen			& 3a. WENN ein Fehler auftritt, DANN wird der Benutzer 
							  benachrichtigt und er kann den Kommentar nochmal eingeben. Es werden keine Daten in der Datenbank abgespeichert.\\ \hline
	Priorität				& hoch \\ \hline
	Verwendungshäufigkeit	& sehr häufig \\ \hline
\end{tabularx}

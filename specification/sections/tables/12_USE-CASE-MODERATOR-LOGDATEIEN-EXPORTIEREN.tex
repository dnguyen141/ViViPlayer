\begin{tabularx}{\linewidth}{|l|X|}
	\hline
	Use Case Nr. 12			& \textbf{Log-Dateien exportieren} \\ \hline
	Erläuterungen			& Nach der Beendigung der ViViPlayer-Session sollte einige Dateien
							  als Ergebnissicherung exportiert werden.\\ \hline
	Systemgrenzen (Scope)	& Gesamtsystem. \\ \hline
	Ebene					& Hauptfunktion. \\ \hline
	Vorbedingung			& Das System ist betriebsbereit. Der Benutzer befindet sich auf der 
							  ViViPlayer-Seite der Web-App. \\ \hline
	Mindestgarantie			& Die Log-Dateien werden mit einer Fehlermeldung exportiert. Die 
							  Daten von den Benutzer, die nicht Fehler haben, werden exportiert. Das Dateisystem vom Server funktioniert einwandfrei.\\ \hline
	Erfolgsfall				& Die exportierten Dateien wurden im Server abgespeichert. Der 
							  Benutzer konnte solche Dateien problemlos herunterladen. \\ \hline
	Stakeholder				& Systembediener (Moderator) - möchte Log-Dateien exportieren 
							  können.\\ 
							& Entwickler - möchten die Log-Dateien zu Trello- oder Jira-Board 
							  importieren können. \\ \hline
	Hauptakteur				& Systembediener (Moderator) \\ \hline
	Auslöser				& Der Benutzer klickt auf das "Session Beenden"-Button. \\ \hline	
	Hauptszenario			& 1. Der Benutzer klickt auf das "Session Beenden"-Button. \\
							& 2. Das System sammelt alle User Stories und Kommentare von den 
							  Benutzern sowie Screenshot des ViVis und speichert die im Server 
							  ab. \\
							& 3. Der Benutzer klickt auf das ``Exportieren''-Button. \\
							& 4. Das System erlaubt dem Benutzer, die Dateien herunterzuladen. \\
							& 5. Der Benutzer speichert die Dateien in seinem System ab. 
							  \\ \hline
	Erweiterungen			& 2a. WENN ein Fehler auftritt, DANN wird die Log-Dateien mit 
							  Fehlermeldung im Server-Dateisystem abgespeichert. Der Benutzer bekommt dann keine URL zum Herunterladen. \\ \hline
	Priorität				& sehr hoch \\ \hline
	Verwendungshäufigkeit	& regelmäßig \\ \hline
\end{tabularx}


\begin{tabularx}{\linewidth}{|l|X|}
	\hline
	Use Case Nr. 12			& \textbf{Dateien exportieren und importieren} \\ \hline
	Erläuterungen			& Nach der Beendigung der ViViPlayer-Session sollten alle User Stories mit 
							  ausgewählten Screenshots exportiert werden, sodass es leicht wieder in ein Trello-Board importiert werden können. \\ \hline
	Systemgrenzen (Scope)	& Gesamtsystem. \\ \hline
	Ebene					& Hauptfunktion. \\ \hline
	Vorbedingung			& Das System ist betriebsbereit. Der Benutzer befindet sich auf der 
							  ViViPlayer-Seite der Web-App. \\ \hline
	Mindestgarantie			& Die Dateien werden mit einer Fehlermeldung exportiert. Die 
							  Daten von den Benutzern, die keine Fehler haben, werden exportiert. Das Dateisystem vom Server funktioniert einwandfrei.\\ \hline
	Erfolgsfall				& Die exportierten Dateien wurden im Server abgespeichert. Der 
							  Benutzer konnte solche Dateien problemlos herunterladen. \\ \hline
	Stakeholder				& Systembediener (Moderator) - möchte Dateien exportieren können. \\ 
							& Entwickler - möchten die Dateien in ein Trello-Board importieren können. \\ \hline
	Hauptakteur				& Systembediener (Moderator) \\ \hline
	Auslöser				& Der Benutzer klickt auf den ``Session Beenden''-Button. \\ \hline	
	Hauptszenario			& 1. Der Benutzer klickt auf den ``Session Beenden''-Button. \\
							& 2. Das System sammelt alle User Stories und Kommentare von den 
							  Benutzern sowie Screenshots des ViVis und speichert die im Server ab. \\
							& 3. Der Benutzer klickt auf den ``Importieren''-Button. \\
							& 4. Das System importiert alle gespeicherte Dateien aus dem Trello-Board des
							  Benutzers. \\ \hline
	Erweiterungen			& 2a. WENN ein Fehler beim Exportieren auftritt, DANN wird die Dateien mit 
							  Fehlermeldung im Server-Dateisystem abgespeichert. \\ 
							& 4a. WENN ein Fehler beim Importieren auftritt, DANN wird eine Fehlermeldung 
							  ausgegeben und keine Dateien nach dem Trello-Board des Benutzers importiert. 
							  \\ \hline
	Priorität				& hoch \\ \hline
	Verwendungshäufigkeit	& häufig \\ \hline
\end{tabularx}


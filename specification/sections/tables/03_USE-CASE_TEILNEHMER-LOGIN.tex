\begin{tabularx}{\linewidth}{|l|X|}
	\hline
	Use Case Nr. 03			& \textbf{Teilnehmer Einloggen} \\ \hline
	Erläuterungen			& Damit ein Teilnehmer alle Funktionen der App benutzen kann, 
							  muss er sein Zugriffsrecht via TAN bestätigen. \\ \hline
	Systemgrenzen (Scope)	& Login-System \\ \hline
	Ebene					& Hauptebene \\ \hline
	Vorbedingung			& Die Web-App ist betriebsbereit. Der Benutzer befindet sich in
							  Hauptseite der Web-App \\ \hline
	Mindestgarantie			& Das Log-in des Benutzers wird abgesagt, falls die eingegebene
							  TAN nicht richtig ist. Eine Fehlermeldung wird ausgegeben
							  \\ \hline
	Erfolgsgarantie			& Der Zugriff des Benutzers wird erfolgreich bestätigt. Der
							  Benutzer wird zur Teilnehmer-Hauptseiten weitergeleitet. 
							  \\ \hline
	Stakeholder				& Der Teilnehmer - möchte die Funktionen der Web-App schnell 
							  wie möglich nutzen. \\
							& Herr Jianwei Shi - möchte die Funktionen der Web-App für die Benutzer 
							  mit korrektem Zugriffrecht verfügbar sein.\\ \hline
	Hauptakteur				& Der Benutzer \\ \hline
	Auslöser				& Der Benutzer möchte in einem Session des ViViPlayer teilnehmen
							  \\ \hline	
	Hauptszenario			& 1. Das System fordert den Benutzer auf, als Teilnehmer eine
							  TAN einzugeben. \\
							& 2. Der Benutzer gibt seine vom Moderator erhaltene TAN ein. \\
							& 3. Das System validiert die vom Benutzer eingegebene
							  TAN. \\
							& 4. Dem Benutzer wird im System angemeldet und weiter zur
							  Teilnehmer-Hauptseite geleitet. \\ \hline
	Erweiterung				& 3a. Wenn das System feststellt, dass die eingegebene TAN nicht 
							  eine von System generierte TAN ist, wird eine Fehlermeldung 
							  ausgegeben. Dem Benutzer wird wieder zurück zu Hauptseite des
							  ViViPlayer geleitet, damit er noch einmal versuchen kann. \\
							  \hline
	Priorität				& mittel \\ \hline
	Verwendungshäufigkeit	& sehr häufig \\ \hline
\end{tabularx}

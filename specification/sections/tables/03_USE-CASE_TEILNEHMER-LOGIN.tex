\begin{tabularx}{\linewidth}{|l|X|}
	\hline
	Use Case Nr. 03			& \textbf{Teilnehmer Einloggen} \\ \hline
	Erläuterungen			& Damit ein Teilnehmer alle Funktionen der App benutzen kann, 
							  muss er sein Zugriffsrecht via TAN bestätigen. \\ \hline
	Systemgrenzen (Scope)	& Login-System \\ \hline
	Ebene					& Hauptebene \\ \hline
	Vorbedingung			& Die Web-App ist betriebsbereit. Der Benutzer befindet sich in
							  Hauptseite der Web-App \\ \hline
	Mindestgarantie			& Das Log-in des Benutzers wird abgesagt, falls die eingegebene
							  TAN nicht richtig ist. Eine Fehlermeldung wird ausgegeben
							  \\ \hline
	Erfolgsfall 			& Der Zugriff des Benutzers wurde erfolgreich bestätigt. Der
							  Benutzer hat sich auf der Moderator-Hauptseite befindet. \\ \hline
	Stakeholder				& Systembediener - möchte die Funktionen der Web-App schnell 
							  wie möglich nutzen. \\
							& Herr Jianwei Shi (Systembesitzer) - möchte die Funktionen der 
							  Web-App für die Benutzer mit korrektem Zugriffsrecht verfügbar sein.\\ \hline
	Hauptakteur				& Systembediener, die Teilnehmer-Rechte haben möchten \\ \hline
	Auslöser				& Der Benutzer möchte an einem Session des ViViPlayer
							  teilnehmen. Er befindet sich auf der Hauptseite der Web-App,
							  um mit TAN einzuloggen. \\ \hline	
	Hauptszenario			& 1. Der Benutzer befindet sich auf der Hauptseite der Web-App,
	                          um mit TAN einzuloggen. \\
	                        & 2. Das System fordert den Benutzer auf, als Teilnehmer eine
							  TAN einzugeben. \\
							& 3. Der Benutzer gibt seine vom Moderator erhaltene TAN ein. \\
							& 4. Das System validiert die vom Benutzer eingegebene
							  TAN. \\
							& 5. Dem Benutzer wird im System angemeldet und weiter zur
							  Teilnehmer-Hauptseite geleitet. \\ \hline
	Erweiterungen			& 4a. Wenn das System feststellt, dass die eingegebene TAN nicht 
							  eine von System generierte TAN ist, wird eine Fehlermeldung 
							  ausgegeben. Das System geht zurück zu Schritt 2. \\ \hline
	Priorität				& mittel \\ \hline
	Verwendungshäufigkeit	& sehr häufig \\ \hline
\end{tabularx}

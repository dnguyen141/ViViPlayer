\begin{tabularx}{\linewidth}{|l|X|}
	\hline
	Use Case Nr. 03			& \textbf{Teilnehmer einloggen} \\ \hline
	Erläuterungen			& Damit ein Teilnehmer alle Funktionen der App benutzen kann, 
							  muss er sein Zugriffsrecht via TAN bestätigen. \\ \hline
	Systemgrenzen (Scope)	& Login-System \\ \hline
	Ebene					& Hauptebene \\ \hline
	Vorbedingung			& Die Web-App ist betriebsbereit. Der Benutzer befindet sich auf
							  Hauptseite der Web-App, um mit TAN einzuloggen \\ \hline
	Mindestgarantie			& Im Fehlerfall wird Das Einloggen vom Benutzer abgesagt, falls die 
							  eingegebene TAN nicht richtig ist. Eine Fehlermeldung wird 
							  ausgegeben. \\ \hline
	Erfolgsfall 			& Der Zugriff des Benutzers wurde erfolgreich bestätigt. Der
							  Benutzer befindet sich auf der Teilnehmer-Hauptseite. \\ \hline
	Stakeholder				& Systembediener (Moderator)- möchte die Funktionen der Web-App 
							  schnell wie möglich nutzen. \\
							& Systembesitzer (Herr Jianwei Shi) - möchte, dass die Funktionen   
							  der Web-App für die Benutzer mit korrektem Zugriffsrecht verfügbar sind.\\ \hline
	Hauptakteur				& Systembediener (Teilnehmer) \\ \hline
	Auslöser				& Der Benutzer klickt auf die ``Einloggen als Teilnehmer''-Option.
							  \\ \hline
	Hauptszenario			& 1. Der Benutzer klickt auf die ``Einloggen als Teilnehmer''-Option.
							  \\
	                        & 2. Das System zeigt das Teilnehmer-Loginsformular an. \\
							& 3. Der Benutzer gibt seine vom Moderator erhaltene TAN ein. \\
							& 4. Das System validiert die vom Benutzer eingegebene
							  TAN. \\
							& 5. Dem Benutzer wird im System angemeldet und weiter zur
							  Teilnehmer-Hauptseite geleitet. \\ \hline
	Erweiterungen			& 4a. WENN das System feststellt, dass die eingegebene TAN nicht 
							  eine vom System generierte TAN ist, DANN wird eine Fehlermeldung 
							  ausgegeben. Zurück zu Schritt 2. \\ \hline
	Priorität				& mittel \\ \hline
	Verwendungshäufigkeit	& sehr häufig \\ \hline
\end{tabularx}

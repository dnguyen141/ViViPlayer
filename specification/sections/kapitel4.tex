\section{Qualitätsanforderungen}

	\subsection{Qualitätsziele des Projekts}
	
	Bei diesem Projekt wird ein großer Fokus auf die Aufrechterhaltung der Softwarequalität in Bezug auf die ISO / IEC 25010-Normen gelegt. 
    \linebreak
    Funktionalität gemäß den in Kapitel 3 enthaltenen funktionalen Anforderungen sollte eine höhe Priorität haben, um Visionsvideos zu erstellen und präsentieren, so wie erfolgreiche Anforderungserhebung zu gewährleisten.
    \linebreak
    Sicherheit und Effizienz sind auch zwei der wichtigsten Qualitätsziele in diesem Projekt. Da die Software private Daten von Dritten, wie beispielsweise Kunden anderer Unternehmen, verarbeiten wird, wird ein großer Fokus auf die Wahrung der Vertraulichkeit, Integrität und Authentifizierung bei allen Benutzer- und Moderatorinteraktionen gelegt. ViViPlayer-Sitzungen sollten nur autorisierten Personen angezeigt werden und alle Sitzungsinformationen, einschließlich User Stories und Umfrageergebnisse, sollten nur dem autorisierten Moderator zugänglich sein. 
    \linebreak
    Da ViViPlayer auf die Live-Interaktion zwischen mehreren Benutzern fokussiert, ist es wichtig, dass auch auf Effizienz großen Wert gelegt wird, damit die Benutzer ein flüssiges Wiedergabeerlebnis (kein Ruckeln) sowie minimale Verzögerungen bei der Synchronisation haben.
    \linebreak
    Im Hinblick auf die Benutzerfreundlichkeit ist es wichtig, dass die Software nicht nur für Moderatoren, sondern auch für Benutzer eine intuitive Benutzererfahrung bietet.  Durch guter Erlernbarkeit und eine intuitive Benutzeroberfläche, sollte ein Benutzer in der Lage sein, auf einen ViViPlayer zuzugreifen und eine Sitzung anzusehen, ohne dass Hilfe vom Moderator erforderlich sind.  Der Moderator sollte in der Lage sein, ein Video hochzuladen und zu segmentieren, ohne nach möglichen Shots zu suchen und bestimmte Zeitstempel eingeben zu müssen. Eine intelligent gestaltete Oberfläche sollte es dem Moderator ermöglichen, eine Live-Sitzung einfach durchzuführen,  Fragen zu stellen und Benutzer-Feedback zu erhalten, ohne Fehler zu machen.
    \linebreak
    \linebreak
    Auch die Wartbarkeit ist für den Kunden sehr wichtig. Es soll möglich sein für eine für neue Entwickler, die Software ohne die Unterstützung der ursprünglichen Entwickler weiterzuentwickeln. Daher ist
    es das Ziel, einen gut strukturierten, kommentierten Code gemäß einer einheitlichen Code Konvention zu haben.
    \linebreak
	
	\subsection{Qualitäts-Prioritäten des Kunden}
		
    Nach Rücksprache mit dem Kunden wurden folgende Prioritäten für die Softwarequalität festgelegt:

		\begin{itemize}
			\item Sicherheit 
            \item Funktionalität
           	\item Benutzbarkeit
            \item Wartbarkeit
            \item Effizienz

		\end{itemize}

	\subsection{Wie Qualitätsziele erreicht werden sollen}
	
	Die Qualität wird während der Entwicklungsphase durch wöchentliche Walkthroughs mit dem Kunden aufrechterhalten. Quality Gates werden auch verwendet, um sicherzustellen, dass die funktionalen Anforderungen mit den höchsten Prioritäten erfüllt sind, bevor man weiter mit neuen Anforderungen anfängt.
	\linebreak
    Durch die Durchführung von Loadtests wird die Effizienz aufrechterhalten. Das Ziel besteht darin, auch bei einer geringeren als erwarteten Kapazität oder einer höheren als erwarteten Ressourcenauslastung eine angemessene Benutzererfahrung zu bieten. 
	\linebreak
	Die Benutzerfreundlichkeit wird durch Walkthroughs mit Standard-Checklisten sowie exemplarische Vorgehensweisen mit Testbenutzern gewährleistet.
	\linebreak
	Die Funktionalität wird durch Black-Box-Tests aufrechterhalten, die durch die Anwendungsfälle in Kapitel 3 bestimmt werden. Außerdem werden White-Box-Tests basierend auf dem Quellcode durchgeführt. Der Test wird durch den gesamten Entwicklungsprozess durch den Einsatz von Komponententests, Integrationstests und Systemtests durchgeführt.
	\linebreak
	Die Sicherheit wird durch die Verwendung von SSL, sicheren Passwörtern/TANs, und Passwort-Hashing gewährleistet. Während der Entwicklung werden auch Penetrationstests durchgeführt.
	\linebreak
	Die Wartbarkeit der Software wird durch die Verwendung einer einheitlichen Code Konvention sowie durch gut kommentierten Code und die Reduzierung der Codekomplexität gewährleistet.
	\linebreak
	Wenn möglich, werden nur gängige Software- und Hardwarekombinationen verwendet, um die Portabilität aufrechtzuerhalten.

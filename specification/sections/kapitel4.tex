\section{Qualitätsanforderungen}

	\subsection{Qualitätsziele des Projekts}
	
	Bei diesem Projekt wird ein großer Fokus auf die Aufrechterhaltung der Softwarequalität in Bezug auf die ISO / IEC 25010-Normen gelegt. Dazu gehören Funktionalität, Effizienz, Kompatibilität, Benutzerfreundlichkeit, Zuverlässigkeit, Sicherheit, Wartbarkeit und Portabilität.
    \linebreak
    Sicherheit und Effizienz sind zwei der wichtigsten Qualitätsziele in diesem Projekt. Da die Software private Daten von Dritten, wie beispielsweise Kunden anderer Unternehmen, verarbeiten wird, wird ein großer Fokus auf die Wahrung der Vertraulichkeit, Integrität und Authentifizierung bei allen Benutzer- und Moderatorinteraktionen gelegt.
    \linebreak
    Da ViViPlayer einen Schwerpunkt auf die Live-Interaktion zwischen mehreren Benutzern legt, ist es wichtig, dass auch auf Effizienz, einschließlich Zeitverhalten, Ressourcenauslastung und Kapazität, großer Wert gelegt wird.
    \linebreak
    Im Hinblick auf die Benutzerfreundlichkeit ist es wichtig, dass die Software nicht nur für Moderatoren, sondern auch für Benutzer eine angemessene Benutzererfahrung bietet. Es ist wichtig, dass Moderatoren die Option für Verknüpfungen und erweiterte Funktionen haben, damit sie die Software im Laufe der Zeit effizienter nutzen können. Da die Benutzer verschiedene technische Hintergründe haben, wird ein Fokus auf die Erlernbarkeit durch ein intuitives Design und hilfreiche Dialoge gelegt.
    \linebreak
    Auch die Wartbarkeit ist für den Kunden sehr wichtig. Da die Software in Kooperation mit der Leibniz Universität Hannover entwickelt wird, ist es wichtig, dass die Software von Studierenden und Lehrenden für zukünftige Lehrveranstaltungen weiterentwickelt werden kann.
	
	\subsection{Qualitäts-Prioritäten des Kunden}
		
    Nach Rücksprache mit dem Kunden wurden folgende Prioritäten für die Softwarequalität festgelegt:

		\begin{itemize}
			\item Sicherheit durch SSL bei allen interaktion von Benutzern und Moderatoren
            \item Sichere Autorisierung durch geeignete Passwörter und TANs.
            \item Funktionell gemäß den in Kapital 3 enthaltenen funktionalen Anforderungen.
            \item Ein passendes Zeitverhalten für alle Meetings.
            \item Die Möglichkeit, Meetings mit mindestens 18 Mitgliedern ohne merklichen Verlust der Systemleistung aufrechtzuerhalten.

		\end{itemize}

	\subsection{Wie Qualitätsziele erreicht werden sollen}
	
	Die Qualität wird während der Entwicklungsphase durch wöchentliche Walkthroughs mit dem Kunden aufrechterhalten. Quality Gates werden auch verwendet, um sicherzustellen, dass alle wesentlichen funktionalen Anforderungen erfüllt sind, bevor nicht wesentliche Anforderungen.
	\linebreak
	Durch die Durchführung von Loadtests wird die Effizienz aufrechterhalten. Das Ziel besteht darin, auch bei einer geringeren als erwarteten Kapazität oder einer höheren als erwarteten Ressourcenauslastung eine angemessene Benutzererfahrung zu bieten.
	\linebreak
	Die Kompatibilität wird durch ständige Tests nicht nur auf Entwicklungsservern, sondern auch auf dem vom Kunden bereitgestellten Server aufrechterhalten.
	\linebreak
	Die Benutzerfreundlichkeit wird durch eine intuitive Benutzeroberfläche, Walkthroughs mit Standard-Checklisten sowie exemplarische Vorgehensweisen mit Testbenutzern gewährleistet.
	\linebreak
	Die Zuverlässigkeit wird durch Black-Box-Tests aufrechterhalten, die durch die Anwendungsfälle in Kapital 3 bestimmt werden. Außerdem werden White-Box-Tests basierend auf dem Quellcode durchgeführt. Der Test wird durch den gesamten Entwicklungsprozess durch den Einsatz von Komponententests, Integrationstests und Systemtests durchgeführt.
	\linebreak
	Die Sicherheit wird durch die Verwendung von SSL, sicheren Passwörtern/TANs, und Passwort-Hashing gewährleistet. Während der Entwicklung werden auch Penetrationstests durchgeführt.
	\linebreak
	Die Wartbarkeit wird durch die Codierungskonventionen gewährleistet, z. B. durch entsprechend kommentierten Code und einen Fokus auf die Minimierung der Codekomplexität und Lines of Code.
	\linebreak
	Wenn möglich, werden gängige Software- und Hardwarekombinationen verwendet, um die Portabilität aufrechtzuerhalten.

\section{Optionen zur Aufwandsreduktion}

	\subsection{Mögliche Abstriche}
	Die folgenden Funktionen der Web-App sind Sonderwünsche des Kunden, d.h. sie sind optionale Erweiterungen von der Web-App.
	\begin{itemize}
		\item Geschriebene Sätze werden automatisch vervollständigt.
		\item der Nutzer bekommt eine Rückmeldung für die geschriebenen Sätze anhand Anleitungen.
		\item Design Patterns für mögliche Erweiterungen, z.B. Android App fürs Schreiben der Anforderungen.
		\item Annotationen im Video.
	\end{itemize}
	
	\subsection{Inkrementelle Arbeit}
		\begin{enumerate}
			\item Iteration:
			\begin{itemize}
				\item Login für alle Benutzer.
				\item Zugang zur Cloud.
				\item Video Segmentierung.
				\item TAN erstellen.
				\item User Story eingeben und abgeben.
				\item Session beenden.
			\end{itemize}
			\item Iteration:
			\begin{itemize}
				\item Registrierung für Moderator Rolle.
				\item Anleitung für User Story.
				\item Automatische Aufnahme des Video Frames für User Story.
				\item Umfragen stellen.
				\item Balkendiagramm für Ergebnis von Fragen.
				\item Text Navigation für Shots des ViVis.
				\item Automatisches Anhalten des Videos beim Wechseln zwischen den Shots.
				\item Session Daten exportieren von Server als .odt Datei.
				\item User Stories exportieren von Server als CSV Datei und Bilddatei.
				\item User Interface verbessern.
			\end{itemize}
			\item Optionale Erweiterungen:
			\begin{itemize}
				\item Video Annotation hinzufügen während des Abspielens.
				\item Automatisches Vervollständigen von Texteingaben.
				\item Automatische Rückmeldungen für geschriebene Anforderungen.
				\item API für Android oder iOS, z.B. fürs Schreiben von Anforderungen mit Handy.
			\end{itemize}
		\end{enumerate}


		
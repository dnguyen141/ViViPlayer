\section{Abnahme-Testfälle}
\begin{enumerate}
	\item \underline{\textbf{Testfall}} \linebreak
	\textbf{Setup:} Der Nutzer hat Internetverbindung und öffnet die Web-App. Er befindet sich jetzt in Hauptmenü. Es gibt 2 Knöpfe, ''Start a Session as Moderator'' und ''Join a Session''.\linebreak
	\textbf{Eingabe:} Der Nutzer klickt an ''Start a Session as Moderator''\linebreak
	\textbf{Ausgabe:} Ein Login Fenster taucht auf.\linebreak
	\textbf{Eingabe:} Der Nutzer gibt seine E-mail und Passwort ein und an ''Login'' klickt.\linebreak
	\textbf{Ausgabe:} Ein Loading Screen wird erscheinen. Nach ein Paar Sekunde hat der Nutzer Zugriff auf dem Cloud und er kann Videos auswählen, um eine Session zu vorbereiten.
	
	\item \underline{\textbf{Testfall}} \linebreak
	\textbf{Setup:} Der Nutzer hat Internetverbindung und öffnet die Web-App. Er befindet sich jetzt in Hauptmenü. Es gibt 2 Knöpfe, ''Start a Session as Moderator'' und ''Join a Session''. \linebreak
	\textbf{Eingabe:} Der Nutzer klickt an ''Start a Session as Moderator'' \linebreak
	\textbf{Ausgabe:} Ein Login Fenster taucht auf.\linebreak
	\textbf{Eingabe:} Der Nutzer hat noch nicht ein Konto erstellt, und einfach an ''Login'' klickt.\linebreak
	\textbf{Ausgabe:} Eine Error Nachricht ''Invalid Login Data.'' erscheint unter dem Eingabefeld.
	
	\item \underline{\textbf{Testfall}} \linebreak
	\textbf{Setup:} Der Nutzer hat Internetverbindung und öffnet die Web-App. Er befindet sich jetzt in Hauptmenü. Es gibt 2 Knöpfe, ''Start a Session as Moderator'' und ''Join a Session''.\linebreak
	\textbf{Eingabe:} Der Nutzer klickt an ''Start a Session as Moderator'' \linebreak
	\textbf{Ausgabe:} Ein Login Fenster taucht auf.\linebreak
	\textbf{Eingabe:} Der Nutzer klickt an ''Register''.\linebreak
	\textbf{Ausgabe:} Der Nutzer wird zu Registrierungsseite weitergeleitet.\linebreak
	\textbf{Eingabe:} Der Nutzer gibt seine E-mailadresse und Passwort ein, und noch einmal sein Passwort bestätigen. Er klickt danach an ''Register''. \linebreak
	\textbf{Ausgabe:} Der Nutzer befindet sich jetzt in Validierungsvorgang. Angenommen wäre, dass die Validierung erfolgreich ist. Nutzer ist gleich zurück zu Hauptmenü weitergeleitet und eine Nachricht ''You can now Login as Moderator'' wird erscheinen.
	
	\item \underline{\textbf{Testfall}} \linebreak
	\textbf{Setup:} Der Nutzer hat Internetverbindung und öffnet die Web-App. Er befindet sich jetzt in Hauptmenü. Es gibt 2 Knöpfe, ''Start a Session as Moderator'' und ''Join a Session''. \linebreak
	\textbf{Eingabe:} Der Nutzer klickt an ''Start a Session as Moderator'' \linebreak
	\textbf{Ausgabe:} Ein Login Fenster taucht auf.\linebreak
	\textbf{Eingabe:} Der Nutzer klickt an ''Register''.\linebreak
	\textbf{Ausgabe:} Der Nutzer wird zu Registrierungsseite weitergeleitet.\linebreak
	\textbf{Eingabe:} Der Nutzer gibt seine E-mailadresse und Passwort ein, und noch einmal sein Passwort bestätigen. Er klickt danach an ''Register''. \linebreak
	\textbf{Ausgabe:} Der Nutzer befindet sich jetzt in Validierungsvorgang. Angenommen wäre, dass die Validierung nicht erfolgreich ist. Nutzer ist gleich zurück zu Hauptmenü weitergeleitet und eine Nachricht ''Registration Failed.'' wird erscheinen.
	
	\item \underline{\textbf{Testfall}} \linebreak
	\textbf{Setup:} Der Nutzer hat Internetverbindung und öffnet die Web-App. Er befindet sich jetzt in Hauptmenü. Es gibt 2 Knöpfe, ''Start a Session as Moderator'' und ''Join a Session''.
	Eine ViViPlayer-Session ist aber noch nicht von dem Moderator angefangen. \linebreak
	\textbf{Eingabe:} Der Nutzer klickt an ''Join a Session.'' \linebreak
	\textbf{Ausgabe:} Ein Login Fenster taucht auf. Es gibt Eingabefeld für TAN.\linebreak
	\textbf{Eingabe:} Der Nutzer gibt irgendeine TAN Nummer ein. \linebreak
	\textbf{Ausgabe:} Eine Fehlermeldung wird auftauchen.
	
	\item \underline{\textbf{Testfall}} \linebreak
	\textbf{Setup:} Der Nutzer hat Internetverbindung und öffnet die Web-App. Er befindet sich jetzt in Hauptmenü. Es gibt 2 Knöpfe, ''Start a Session as Moderator'' und ''Join a Session''. Angenommen wäre, dass Eine ViViPlayer-Session von dem Moderator angefangen ist und der Nutzer hat schon eine TAN Nummer bekommen. \linebreak
	\textbf{Eingabe:} Der Nutzer klickt an ''Join a Session.'' \linebreak
	\textbf{Ausgabe:} Ein Login Fenster taucht auf. Es gibt Eingabefeld für TAN.\linebreak
	\textbf{Eingabe:} Der Nutzer gibt gegebene TAN Nummer ein. \linebreak
	\textbf{Ausgabe:} Der Nutzer bekommt eine Bestätigungsnachricht und er wird zu der ViViPlayer-Session weitergeleitet als Teilnehmer.
	
	\item \underline{\textbf{Testfall}} \linebreak
	\textbf{Setup:} Der Nutzer hat Internetverbindung und öffnet die Web-App. Er meldet sich als Moderator an. Der Nutzer befindet sich jetzt in dem ViViPlayer Hauptmenu, die mit dem Server verbunden ist. Es gibt Vision Videos die in dem Server gespeichert sind.\linebreak
	\textbf{Eingabe:} Der Nutzer klickt an einem Video und danach an "Continue". \linebreak
	\textbf{Ausgabe:} Der Nutzer wird auf die Videobearbeitungsseite weitergeleitet. 
	
	\item \underline{\textbf{Testfall}} \linebreak
	\textbf{Setup:} Der Nutzer hat Internetverbindung und öffnet die Web-App. Er meldet sich als Moderator an. Der Nutzer befindet sich jetzt in dem ViViPlayer Hauptmenu, die mit dem Server verbunden ist. Es gibt Vision Videos die in dem Server gespeichert sind.\linebreak
	\textbf{Eingabe:} Der Nutzer gibt ein Video URL in Eingabefeld ein. Er klickt danach an ''Submit''. \linebreak
	\textbf{Ausgabe:} Das System überprüft, ob das URL funktioniert. Angenommen wäre, dass das URL funktioniert. \linebreak
	\textbf{Eingabe:} Der Nutzer klickt an ''Continue''.\linebreak
	\textbf{Ausgabe:} Der Nutzer wird auf die Videobearbeitungsseite weitergeleitet.
	
	\item \underline{\textbf{Testfall}} \linebreak
	\textbf{Setup:} Der Nutzer hat Internetverbindung und öffnet die Web-App. Er meldet sich als Moderator an. Der Nutzer befindet sich jetzt in dem ViViPlayer Hauptmenu, die mit dem Server verbunden ist. Es gibt Vision Videos die in dem Server gespeichert sind.\linebreak
	\textbf{Eingabe:} Der Nutzer gibt ein Video URL in Eingabefeld ein. Er klickt danach an ''Submit''. \linebreak
	\textbf{Ausgabe:} Das System überprüft, ob das URL funktioniert. Angenommen wäre, dass das URL nicht funktioniert. \linebreak
	\textbf{Ausgabe:} Eine Fehlermeldung wird gezeigt.
	
	\item \underline{\textbf{Testfall}} \linebreak
	\textbf{Setup:} Der Nutzer hat Internetverbindung und öffnet die Web-App. Er meldet sich als Moderator an. Der Nutzer befindet sich jetzt in dem ViViPlayer Hauptmenu, die mit dem Server verbunden ist. Es gibt Vision Videos die in dem Server gespeichert sind.\linebreak
	\textbf{Eingabe:} Der Nutzer klickt an ''Continue'', aber er hat kein Video ausgewählt. \linebreak
	\textbf{Ausgabe:} Eine Fehlermeldung wird gezeigt.
	
	\item \underline{\textbf{Testfall}} \linebreak
	\textbf{Setup:} Der Nutzer hat Internetverbindung und öffnet die Web-App. Er meldet sich als Moderator an. Der Nutzer hat ein Video ausgewählt, und jetzt ist er auf der Videobearbeitungsseite.\linebreak
	\textbf{Eingabe:} Der Nutzer klickt an Video Slider, um ein bestimmtes Shot auszuwählen. Er gibt den Titel für das Shot ein, und klickt an ''Add Shot''. \linebreak
	\textbf{Ausgabe:} Ein Zeitstemple für das ausgewahlte Shot wird in dem Slider hinzugefügt.\linebreak
	\textbf{Eingabe:} Der Nutzer klickt an ''Continue''. \linebreak
	\textbf{Ausgabe:} Der Nutzer wird zur nächsten Seite weitergeleitet. Eine Benutzeroberfläche für das Einfügen von Annotationen wird gezeigt.
	
	\item \underline{\textbf{Testfall}} \linebreak
	\textbf{Setup:} Der Nutzer hat Internetverbindung und öffnet die Web-App. Er meldet sich als Moderator an. Der Nutzer hat ein Video ausgewählt, und jetzt ist er auf der Videobearbeitungsseite.\linebreak
	\textbf{Eingabe:} Der Nutzer klickt an Video Slider, um ein bestimmtes Shot auszuwählen. Er gibt kein Titel für das Shot ein, und klickt an ''Add Shot''. \linebreak
	\textbf{Ausgabe:} Eine Fehlermeldung wird gezeigt und kein neues Shot wird hinzugefügt.
	
	\item \underline{\textbf{Testfall}} \linebreak
	\textbf{Setup:} Der Nutzer hat Internetverbindung und öffnet die Web-App. Er meldet sich als Moderator an. Der Nutzer hat ein Video ausgewählt, und hat das Video in Shots segmentiert. Er befindet sich jetzt auf der nächsten Seite, wo man Annotationen in dem Video hinzufügen kann.\linebreak
	\textbf{Eingabe:} Der Nutzer wählt ein Shot, und gibt seine Frage in einem Eingabefeld neben dem Player ein. Er gibt auch die Auswähle für die Antwort ein. Danach klickt er an ''Add Annotation.''\linebreak
	\textbf{Ausgabe:} Die Annotation wird in der Mitte des ViViPlayers hinzugefügt.\linebreak
	\textbf{Eingabe:} Der Nutzer klickt an die Annotation, und zieht es irgendwo in dem Shot.\linebreak
	\textbf{Ausgabe:} Die Annotation ist jetzt in einer neuen Position in dem Shot.\linebreak
	\textbf{Eingabe:} Der Nutzer klickt an ''Continue''. \linebreak
	\textbf{Ausgabe:} Der Nutzer wird zur nächsten Seite weitergeleitet. Eine Benutzeroberfläche für die Erstellung von TAN wird gezeigt.
	
	\item \underline{\textbf{Testfall}} \linebreak
	\textbf{Setup:} Der Nutzer hat Internetverbindung und öffnet die Web-App. Er meldet sich als Moderator an. Der Nutzer hat ein Video ausgewählt, und hat das Video in Shots segmentiert. Er befindet sich jetzt auf der nächsten Seite, wo man Annotationen in dem Video hinzufügen kann.\linebreak
	\textbf{Eingabe:} Der Nutzer wählt ein Shot, und gibt seine Frage in einem Eingabefeld neben dem Player ein. Er gibt keine Auswähle für die Antwort ein. Danach klickt er an ''Add Annotation.''\linebreak
	\textbf{Ausgabe:} Eine Fehlermeldung wird gezeigt und keine neue Annotation wird hinzugefügt.
	
	\item \underline{\textbf{Testfall}} \linebreak
	\textbf{Setup:} Der Nutzer hat Internetverbindung und öffnet die Web-App. Er meldet sich als Moderator an. Der Nutzer hat ein Video ausgewählt, das Video in Shots segmentiert und Annotationen eingebettet. Er befindet sich jetzt auf der TAN-Erstellungsseite.\linebreak
	\textbf{Eingabe:} Der Nutzer gibt eine sichere TAN ein.\linebreak
	\textbf{Ausgabe:} Das System überprüft die eingegebene TAN. Angenommen wäre, dass die TAN sicher ist. Eine Nachricht ''TAN is safe.'' wird gezeigt automatisch.\linebreak
	\textbf{Eingabe:} Der Nutzer klickt an ''Continue''.\linebreak
	\textbf{Ausgabe:} Der Nutzer wird auf der nächsten Seite weitergeleitet und ViVi-Überblick-Oberfläche wird von dem System gezeigt.
	
	\item \underline{\textbf{Testfall}} \linebreak
	\textbf{Setup:} Der Nutzer hat Internetverbindung und öffnet die Web-App. Er meldet sich als Moderator an. Der Nutzer hat ein Video ausgewählt, das Video in Shots segmentiert und Annotationen eingebettet. Er befindet sich jetzt auf der TAN-Erstellungsseite.\linebreak
	\textbf{Eingabe:} Der Nutzer gibt TAN ein.\linebreak
	\textbf{Ausgabe:} Das System überprüft die eingegebene TAN. Angenommen wäre, dass die TAN nicht sicher ist. Eine Nachricht ''TAN is not safe.'' wird gezeigt automatisch.\linebreak
	\textbf{Eingabe:} Der Nutzer klickt an ''Continue''.\linebreak
	\textbf{Ausgabe:} Eine Fehlermeldung wird gezeigt und die Session könnte nicht gestartet werden.
	
	\item \underline{\textbf{Testfall}} \linebreak
	\textbf{Setup:} Der Nutzer hat Internetverbindung und öffnet die Web-App. Er meldet sich als Moderator an. Der Nutzer hat ein Video ausgewählt, das Video in Shots segmentiert und Annotationen eingebettet. Er befindet sich jetzt auf der TAN-Erstellungsseite.\linebreak
	\textbf{Eingabe:} Der Nutzer klickt an ''Generate TAN''.\linebreak
	\textbf{Ausgabe:} Das System überprüft die generierte TAN.\linebreak
	\textbf{Eingabe:} Der Nutzer klickt an ''Continue''.\linebreak
	\textbf{Ausgabe:} Der Nutzer wird auf der nächsten Seite weitergeleitet und ViVi-Überblick-Oberfläche wird von dem System gezeigt.
	
\end{enumerate}

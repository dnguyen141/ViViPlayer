\section{Abnahme-Testfälle}
\begin{enumerate}
	\item \underline{\textbf{Testfall: UC 1 Moderator registrieren}} \linebreak
	\textbf{Setup:} Der Benutzer ist mit dem Internet verbunden und die Web-App ist geöffnet. Er befindet sich jetzt auf der Hauptseite.\linebreak
	\textbf{Eingabe:} Der Benutzer klickt auf ``Registrieren.''.\linebreak
	\textbf{Ausgabe:} Der Benutzer wird zur Registrierungsseite weitergeleitet.\linebreak
	\textbf{Eingabe:} Der Benutzer gibt seinen erwünschten Namen ``jianwei.shi'' und Passwort ``Xo73\#\&CP'' ein und bestätigt dieses. Er klickt danach auf ``Registrieren''. \linebreak
	\textbf{Ausgabe:} Die Daten sind zur Validierung zu einem Moderator weitergeleitet worden. Der Nutzer wird zur Hauptseite weitergeleitet und da kann er ein Meldung sehen, dass die Registrierung erfolgreich abgeschlossen ist.
	
	\item \underline{\textbf{Testfall: UC 2 Moderator einloggen}} \linebreak
	\textbf{Setup:} Der Benutzer ist mit dem Internet verbunden und die Web-App ist geöffnet. Er befindet sich jetzt auf der Hauptseite. \linebreak
	\textbf{Eingabe:} Der Benutzer klickt auf ``Einloggen als Moderator''. \linebreak
	\textbf{Ausgabe:} Ein Moderator-Loginformular taucht auf.\linebreak
	\textbf{Eingabe:} Der Benutzer hat keine Anmeldedaten eingegeben und klickt auf ``Einloggen als Moderator''.\linebreak
	\textbf{Ausgabe:} Die Fehlermeldung ``Ungültige Anmeldedaten'' erscheint unter dem Eingabefeld.
	
	\item \underline{\textbf{Testfall: UC 2 Moderator einloggen}} \linebreak
	\textbf{Setup:} Der Benutzer ist mit dem Internet verbunden und die Web-App ist geöffnet. Er befindet sich jetzt auf der Hauptseite. Der Benutzername ``jianwei.shi'' und das Passwort ``Xo73\#\&CP'' sind im System hinterlegt. \linebreak
	\textbf{Eingabe:} Der Benutzer klickt auf ``Einloggen als Moderator''. \linebreak
	\textbf{Ausgabe:} Ein Moderator-Loginformular taucht auf.\linebreak
	\textbf{Eingabe:} Der Benutzer gibt die hinterlegten Anmeldedaten ein und klickt auf ``Einloggen als Moderator''.\linebreak
	\textbf{Ausgabe:} Der Benutzer ist als Moderator angemeldet und wird zur Moderator-Hauptseite geleitet.
	
	\item \underline{\textbf{Testfall: UC 3 Teilnehmer einloggen}} \linebreak
	\textbf{Setup:} Der Benutzer ist mit dem Internet verbunden und die Web-App ist geöffnet. Er befindet sich jetzt auf der Hauptseite. Eine ViViPlayer-Session ist von dem Moderator gestartet worden und im System ist die TAN ``y6X2\$R'' hinterlegt.
	Eine ViViPlayer-Session ist aber noch nicht von dem Moderator gestartet. \linebreak
	\textbf{Eingabe:} Der Benutzer klickt auf ``Einloggen als Teilnehmer''. \linebreak
	\textbf{Ausgabe:} Ein Einloggen-Fenster taucht auf. \linebreak
	\textbf{Eingabe:} Der Benutzer gibt die TAN ``y6X2R\$'' ein. \linebreak
	\textbf{Ausgabe:} Eine Fehlermeldung wird angezeigt.
	
	\item \underline{\textbf{Testfall: UC 3 Teilnehmer einloggen}} \linebreak
	\textbf{Setup:} Der Benutzer ist mit dem Internet verbunden und die Web-App ist geöffnet. Er befindet sich jetzt auf der Hauptseite. Eine ViViPlayer-Session ist von dem Moderator gestartet worden und im System ist die TAN ``5j\$M06'' hinterlegt. \linebreak
	\textbf{Eingabe:} Der Benutzer klickt auf ``Einloggen als Teilnehmer''. \linebreak
	\textbf{Ausgabe:} Ein Einloggen-Fenster taucht auf. \linebreak
	\textbf{Eingabe:} Der Benutzer gibt die hinterlegte TAN ein. \linebreak
	\textbf{Ausgabe:} Der Benutzer bekommt eine Bestätigungsnachricht und er wird zu der ViViPlayer-Session weitergeleitet als Teilnehmer.

	\item \underline{\textbf{Testfall: UC 4 Session vorbereiten}} \linebreak
	\textbf{Setup:} Der Benutzer ist mit dem Internet verbunden und die Web-App ist geöffnet. Er ist als Moderator angemeldet und befindet sich jetzt auf der Moderator-Hauptseite (Testfall UC 2 Moderator einloggen).\linebreak
	\textbf{Eingabe:} Der Benutzer klickt auf ``Session erstellen''.\linebreak
	\textbf{Ausgabe:} Das System zeigt die ViVi-Auswählen-Oberfläche an. \\
	\textbf{Eingabe:} Der Benutzer wählt ein ViVi und bestätigt (Testfall UC 5 ViVi auswählen).\\
	\textbf{Ausgabe:} Das System zeigt die ViVi-Segmentieren-Oberfläche an. \\
	\textbf{Eingabe:} Der Benutzer segmentiert das von ihm hochgeladene Video und bestätigt. (Testfall UC 6 ViVi Segmentieren)\\
	\textbf{Ausgabe:} Das System zeigt die Frage-Vorbereiten-Oberfläche an.\\
	\textbf{Eingabe:} Der Benutzer bereitet eine Umfrage/Verständnisfrage vor und bestätigt. (Testfall UC 7 Umfrage/Verständnisfragen erstellen)\\
	\textbf{Ausgabe:} Das System zeigt die ViVi-Überblick-Oberfläche an.\\
	\textbf{Eingabe:} Der Benutzer prüft noch mal alle Vorbereitungsschritte und klickt auf ``Erstellen''.\\
	\textbf{Ausgabe:} Das System erstellt eine neue Session und eine sichere TAN.
	
	\item \underline{\textbf{Testfall: UC 5 ViVi auswählen}} \linebreak
	\textbf{Setup:} Der Benutzer ist mit dem Internet verbunden und die Web-App ist geöffnet. Er ist als Moderator angemeldet. Er ist im Vorgang, eine Session zu erstellen (Testfall UC 4 Session vorbereiten). Er hat ein Vision-Video in seinem Dateisystem gespeichert. \linebreak
	\textbf{Eingabe:} Der Benutzer klickt auf ``Hochladen''. \linebreak
	\textbf{Ausgabe:} Das System zeigt ein Fenster zum Hochladen des ViVis an.\\
	\textbf{Eingabe:} Der Benutzer wählt ein ViVi von seinem Dateisystem aus und bestätigt.\\
	\textbf{Ausgabe:} Das System zeigt die ViVi-Segmentieren-Oberfläche an.
	
	\item \underline{\textbf{Testfall: UC 6 ViVi segmentieren}} \linebreak
	\textbf{Setup:} Der Benutzer ist mit dem Internet verbunden und die Web-App ist geöffnet. Er ist als Moderator angemeldet. Er ist im Vorgang, eine Session zu erstellen (Testfall UC 4 Session vorbereiten). Er hat ein Video ausgewählt. Er befindet sich jetzt auf der ViVi-Segmentieren-Oberfläche.\linebreak
	\textbf{Eingabe:} Der Benutzer klickt auf den Video Slider, um einen bestimmten Shot auszuwählen. Er gibt den Titel für den Shot ein und klickt auf ``Hinzufügen''. \linebreak
	\textbf{Ausgabe:} Ein Zeitstempel für den ausgewählten Shot wird dem Slider hinzugefügt.\linebreak
	\textbf{Eingabe:} Der Benutzer klickt auf ``Weiter''. \linebreak
	\textbf{Ausgabe:} Der Benutzer wird zur nächsten Seite weitergeleitet. Eine Benutzeroberfläche zum Einfügen von Umfrage/Verständnisfragen wird gezeigt.
	
	\item \underline{\textbf{Testfall: UC 6 ViVi segmentieren}} \linebreak
	\textbf{Setup:} Der Benutzer ist mit dem Internet verbunden und die Web-App ist geöffnet. Er ist als Moderator angemeldet. Er ist im Vorgang, eine Session zu erstellen (Testfall UC 4 Session vorbereiten). Er hat ein Video ausgewählt und befindet sich jetzt auf der ViVi-Segmentieren-Oberfläche.\linebreak
	\textbf{Eingabe:} Der Benutzer klickt auf den Video-Slider, um einen bestimmten Shot auszuwählen. Er gibt keinen Titel für den Shot ein und klickt auf ``Hinzufügen''. \linebreak
	\textbf{Ausgabe:} Eine Fehlermeldung wird gezeigt und kein neuer Shot wird hinzugefügt.
	
	\item \underline{\textbf{Testfall: UC 7 Umfragen und Verständnisfragen erstellen}} \linebreak
	\textbf{Setup:} Der Benutzer ist mit dem Internet verbunden und die Web-App ist geöffnet. Er ist als Moderator angemeldet. Er ist im Vorgang, eine Session zu erstellen (Testfall UC 4 Session vorbereiten). Er hat ein Video ausgewählt und das Video in Shots segmentiert und befindet sich jetzt auf der Frage-Vorbereiten-Oberfläche.\\
	\textbf{Eingabe:} Der Benutzer klickt auf ``Weiter''. \\
	\textbf{Ausgabe:} Das System zeigt die Frage-Vorbereiten-Oberfläche.\\ 
	\textbf{Eingabe:} Der Benutzer gibt die Frage ``Kann ein Moderator eine User Story verfassen?'' ein. Er legt fest, dass es sich um eine Verständnisfrage handelt und gibt die Antwortmöglichtkeiten ``Ja'' und ``Nein'' an. ``Ja'' wird als richtige Antwort festgelegt.\\
	\textbf{Ausgabe:} Das System fügt die neue Verständnis-/Umfrage in der Liste hinzu. \\
	\textbf{Eingabe:} Der Benutzer klickt auf ``Weiter''.\\
	\textbf{Ausgabe:} Das System zeigt die ViVi-Überblick-Oberfläche an (Testfall UC 4 Session vorbereiten).

	\item \underline{\textbf{Testfall: UC 8 ViVi steuern}} \linebreak
	\textbf{Setup:} Der Benutzer ist mit dem Internet verbunden und die Web-App ist geöffnet. Er ist als Moderator angemeldet. Eine ViViPlayer-Session ist von dem Benutzer gestartet worden (Testfall UC 4 Session vorbereiten). Ein weiterer Benutzer ist als Teilnehmer in der Session angemeldet. \\
	\textbf{Eingabe:} Der Benutzer bewegt den Slider an Stelle 01:40. \\
	\textbf{Ausgabe:} Das Video spielt an der Stelle 01:40 weiter für alle Teilnehmer.\\
	
	\item \underline{\textbf{Testfall: UC 9 Umfrage oder Verständnisfrage in Session stellen}} \linebreak
	\textbf{Setup:} Der Benutzer ist mit dem Internet verbunden und die Web-App ist geöffnet. Er ist als Moderator angemeldet. Es ist ein Video auf dem Server gespeichert. Der Benutzer hat ein Video ausgewählt und das Video in Shots segmentiert. Es gibt einen weiteren Teilnehmer, der angemeldet und Teil der Session ist.\\
	\textbf{Eingabe:} Der Benutzer klickt auf ``Frage'' Tab. \\
	\textbf{Ausgabe:} Das System zeigt das Eingabefeld an.\\ 
	\textbf{Eingabe:} Der Benutzer gibt die Frage ``Kann ein Moderator eine User Story verfassen?''. Er legt fest, dass es sich um eine Verständnisfrage handelt und gibt die Antwortmöglichtkeiten ``Ja'' und ``Nein'' an. ``Ja'' wird als richtige Antwort festgelegt.\\
	\textbf{Ausgabe:} Das System zeigt die Frage allen Teilnehmern an.\\ 
	\textbf{Eingabe:} Alle Benutzer antworten auf die Umfrage. \\
	\textbf{Ausgabe:} Das System zeigt jedem Benutzer nach der Bestätigung des Moderators das Ergebnis an. Die Verständnisfrage wird gespeichert und dem dazugehörigen Shot zugeordnet. \\
	
	\item \underline{\textbf{Testfall: UC 10 User-Story schreiben}} \linebreak
	\textbf{Setup:} Der Benutzer ist mit dem Internet verbunden und die Web-App ist geöffnet. Der Benutzer ist angemeldet und Teil der Session.\\
	\textbf{Eingabe:} Der Benutzer wählt den ``User Story''-Modus aus und gibt ``Damit ich weiß, ob jemand unregelmäßig arbeitet, möchte ich als Abteilungsleiter eine visuelle Darstellung der geleisteten Stunden sehen.'' und bestätigt.\\
	\textbf{Ausgabe:} Das System verarbeitet die Anfrage und speichert die User Story ab. Die User Story wird den anderen Benutzern angezeigt und dem dazugehörigen Shot zugeordnet.\linebreak \linebreak
	
	\item \underline{\textbf{Testfall: UC 11 In Session kommentieren}} \linebreak
	\textbf{Setup:} Der Benutzer ist mit dem Internet verbunden und die Web-App ist geöffnet. Er ist angemeldet und Teil der Session.\\
	\textbf{Eingabe:} Der Benutzer wählt den ``Kommentar''-Modus aus. \\
	\textbf{Ausgabe:} Das System zeigt die ``Kommentar''-Eingabe. \\
	\textbf{Eingabe:} Der Benutzer gibt ``Als ein Mitarbeiter möchte ich nicht, dass meine private 
	Arbeit für alle sichtbar ist.'' ein und bestätigt.\\
	\textbf{Ausgabe:} Das System verarbeitet die Anfrage und speichert den Kommentar ab. Die User Story wird den anderen Benutzern angezeigt und dem dazugehörigen Shot zugeordnet.
	
	\item \underline{\textbf{Testfall: UC 12 Dateien exportieren}} \linebreak
	\textbf{Setup:} Der Benutzer ist mit dem Internet verbunden und die Web-App ist geöffnet. Er ist als Moderator angemeldet. Der Benutzer befindet sich jetzt auf der ViViPlayer-Seite. \\
	\textbf{Eingabe:} Der Benutzer klickt auf ``Session Beenden''.\\
	\textbf{Ausgabe:} Session ist beendet. Der Kommentar ``Als ein Mitarbeiter möchte ich nicht, dass meine private Arbeit für alle sichtbar ist.'', die Verständnisfrage ``Kann der Manager neue Aufgaben für die Mitarbeiter zuweisen?'' mit den Antworten ``Ja'' und ``Nein'' und die Anzahl der Teilnehmer, die ``Ja'' bzw. ``Nein'' für jede Verständnisfrage geantwortet haben, sind auf dem Server als ``.odt''-Datei gespeichert. Die User Story ``Damit ich weiß, ob jemand unregelmäßig arbeitet, möchte ich als Abteilungsleiter eine visuelle Darstellung der geleisteten Stunden sehen.'', die User Story ``Damit wir diese Web-App gemeinsam nutzen können, möchte ich meine Kollegen einladen.'' und die User Story ``Damit ich über unsere Erfolge und Fehlschläge besser berichten kann, möchte ich als Manager die Fortschritte meiner Kollegen nachvollziehen können.'' sind auf dem Server als ``.csv''-Datei gespeichert. Alle Screenshots sind auf dem Server als Bilddateien gespeichert.\\
	\textbf{Eingabe:} Der Benutzer klickt auf ``Exportieren''. \\
	\textbf{Ausgabe:} Das System zeigt ein Pop-Up-Fenster, wo der Benutzer die Speicherort der exportierten Dateien in seinem Dateisystem auswählen und bestätigen kann.\\
	\textbf{Eingabe:} Der Benutzer wählt die gewünschte Speicherort und bestätigt. \\
	\textbf{Ausgabe:} Das System lädt die exportierten Dateien unter uns speichert in dem Benutzers Dateisystem. \\ 
	
	\item \underline{\textbf{Testfall: UC 13 An Umfrage oder Verständnisfrage teilnehmen}} \linebreak
	\textbf{Setup:} Der Benutzer ist mit dem Internet verbunden und die Web-App ist geöffnet. Er ist als Teilnehmer angemeldet und Teil der Session. Er befindet sich jetzt auf der ViViPlayer-Seite der Web-App. Der Moderator hat eine Umfrage/Verständnisfrage gestellt (Testfall UC 9 Umfrage/Verständnisfragen in Session stellen).\\
	\textbf{Eingabe:} Der Benutzer klickt auf eine Antwort der Umfrage/Verständnisfrage und auf ``Senden''.\\
	\textbf{Ausgabe:} Das System sammelt die Antworten von allen Benutzern und nach der Bestätigung des Moderators zeigt es diese in Form eines Balkendiagramms an.
	
\end{enumerate}

\section{Funktionale Anforderungen}

\subsection{Use Case-Diagramm}
\includegraphics[width=\linewidth]{sections/diagrams/viviplayer-uc-diagramm.png}
\pagebreak

\subsection{Use Case-Beschreibungen}

\subsubsection{UC: Moderator registrieren}
\begin{tabularx}{\linewidth}{|l|X|}
	\hline
	Use Case Nr. 01			& \textbf{Moderator registrieren} \\ \hline
	Erläuterungen			& Um alle Rechte eines Moderators zu bekommen, muss ein 
							Benutzer als Moderator registriert werden. \\ \hline
	Systemgrenzen (Scope)	& Registrierung-System. \\ \hline
	Ebene					& Hauptfunktion \\ \hline
	Vorbedingung			& Die Web-App ist betriebsbereit. Der Benutzer befindet sich auf die 
							  Hauptseite von ViViPlayer-Web-App. \\ \hline
	Mindestgarantie			& Im Fehlerfall wird kein neues Moderator-Konto im System 
	                          registriert. Klare Fehlermeldung wird ausgegeben. \\ \hline
	Erfolgsfall  			& Ein neues Moderator-Konto wurde erstellt. \\ \hline
	Stakeholder				& Systembediener (Moderator) - möchte die Funktionen der Web-App so 
							  schnell wie möglich nutzen. \\
							& Systembesitzer (Systemadministrator) - möchte, dass die Funktionen 
							  der Web-App für die Benutzer mit korrekten Zugriffsrechten verfügbar sind.\\ \hline
	Hauptakteur				& Systembediener (Moderator) \\ \hline
	Auslöser				& Der Benutzer klickt auf den ``Registrieren''-Button. \\ \hline	
	Hauptszenario			& 1. Der Benutzer klickt auf den ``Registrieren''-Button. \\
							& 2. Das System fordert den Benutzer auf, eine Email-Adresse 
							  sowie die Passwort für das neue Konto einzugeben. \\
							& 3. Der Benutzer gibt seinen Benutzername und seine Passwort 
							  ein und bestätigt sie \\
							& 4. Das System und ein anderer Moderator validiert die vom 
							  Benutzer eingegebene Daten. Eine Meldung wird als 
							  Anleitung angezeigt, was der neue Benutzer weiter machen soll. \\
							& 5. Der Benutzer wird zurück zur Hauptseite geleitet. 
							  \\ \hline
	Erweiterungen			& 4a. WENN das System feststellt, dass die eingegebene Daten 
							  (Benutzername und Passwort) nicht gültig sind, DANN wird eine Fehlermeldung ausgegeben. Zurück zu Schritt 2. \\ \hline
	Priorität				& niedrig \\ \hline
	Verwendungshäufigkeit	& weniger häufig \\ \hline
\end{tabularx}

\pagebreak

\subsubsection{UC: Moderator einloggen}
\begin{tabularx}{\linewidth}{|l|X|}
	\hline
	Use Case Nr. 02			& \textbf{Moderator Einloggen} \\ \hline
	Erläuterungen			& Als ein ScrumMaster/Projektleiter möchte der Benutzer einloggen und
							  damit er eine Session im ViViPlayer abhalten kann.  
							  \\ \hline
	Systemgrenzen (Scope)	& Login-System \\ \hline
	Ebene					& Hauptebene \\ \hline
	Vorbedingung			& Die Web-App ist betriebsbereit. Der Benutzer befindet sich in
							  Hauptseite der Web-App \\ \hline
	Mindestgarantie			& Das Log-in des Benutzers wird abgesagt, falls der eingegebene
							  Benutzername/die Email-Adresse nicht im Datenbank oder die Passwort falsch ist. Eine Fehlermeldung wird auf jedem Fall ausgegeben.
							  \\ \hline
	Erfolgsgarantie			& Der Zugriff des Benutzers wird erfolgreich bestätigt. Der
							  Benutzer wird zur Moderator-Hauptseiten weitergeleitet. 
							  \\ \hline
	Stakeholder				& Moderator - möchte die Funktionen der Web-App schnell 
							  wie möglich nutzen. \\
							& Herr Jianwei Shi - möchte die Funktionen der Web-App für die Benutzer 
							  mit korrektem Zugriffrecht  verfügbar sein.\\ \hline
	Hauptakteur				& Der Benutzer \\ \hline
	Auslöser				& Der Benutzer möchte als Moderator einloggen.
							  \\ \hline	
	Hauptszenario			& 1. Das System fordert den Benutzer auf, als Moderator ein
							  Benutzername/eine Email-Adresse sowie die Passwort einzugeben. \\
							& 2. Der Benutzer gibt sein Benutzername/seine Email-Adresse und seine
							  Passwort ein. \\
							& 3. Das System validiert die vom Benutzer eingegebene
							  Log-in-Daten. \\
							& 4. Dem Benutzer wird im System angemeldet und weiter zur
							  Moderator-Hauptseite geleitet. \\ \hline
	Erweiterung				& 3a. Wenn das System feststellt, dass die eingegebene Daten (Benutzername/
							  Email-Adresse und Passwort) nicht abgestimmt sind, wird eine Fehlermeldung 
							  ausgegeben. Der Benutzer wird dann zurück zur Hauptseite geleitet, damit
							  er noch einmal versuchen kann.
							  \\ \hline
	Priorität				& mittel \\ \hline
	Verwendungshäufigkeit	& häufig \\ \hline
\end{tabularx}

\pagebreak

\subsubsection{UC: Teilnehmer einloggen}
\begin{tabularx}{\linewidth}{|l|X|}
	\hline
	Use Case Nr. 03			& \textbf{Teilnehmer einloggen} \\ \hline
	Erläuterungen			& Damit ein Teilnehmer alle Funktionen der App benutzen kann, 
							  muss er sein Zugriffsrecht via TAN bestätigen. \\ \hline
	Systemgrenzen (Scope)	& Login-System \\ \hline
	Ebene					& Hauptebene \\ \hline
	Vorbedingung			& Die Web-App ist betriebsbereit. Der Benutzer befindet sich auf
							  Hauptseite der Web-App, um mit TAN einzuloggen \\ \hline
	Mindestgarantie			& Im Fehlerfall wird Das Einloggen vom Benutzer abgesagt, falls die 
							  eingegebene TAN nicht richtig ist. Eine Fehlermeldung wird 
							  ausgegeben. \\ \hline
	Erfolgsfall 			& Der Zugriff des Benutzers wurde erfolgreich bestätigt. Der
							  Benutzer befindet sich auf der Teilnehmer-Hauptseite. \\ \hline
	Stakeholder				& Systembediener (Moderator)- möchte die Funktionen der Web-App 
							  schnell wie möglich nutzen. \\
							& Systembesitzer (Systemadministrator) - möchte, dass die Funktionen   
							  der Web-App für die Benutzer mit korrektem Zugriffsrecht verfügbar sind.\\ \hline
	Hauptakteur				& Systembediener (Teilnehmer) \\ \hline
	Auslöser				& Der Benutzer klickt auf die ``Einloggen als Teilnehmer''-Option.
							  \\ \hline
	Hauptszenario			& 1. Der Benutzer klickt auf die ``Einloggen als Teilnehmer''-Option.
							  \\
	                        & 2. Das System zeigt das Teilnehmer-Loginsformular an. \\
							& 3. Der Benutzer gibt seine vom Moderator erhaltene TAN ein. \\
							& 4. Das System validiert die vom Benutzer eingegebene
							  TAN. \\
							& 5. Dem Benutzer wird im System angemeldet und weiter zur
							  Teilnehmer-Hauptseite geleitet. \\ \hline
	Erweiterungen			& 4a. WENN das System feststellt, dass die eingegebene TAN nicht 
							  eine vom System generierte TAN ist, DANN wird eine Fehlermeldung 
							  ausgegeben. Zurück zu Schritt 2. \\ \hline
	Priorität				& mittel \\ \hline
	Verwendungshäufigkeit	& sehr häufig \\ \hline
\end{tabularx}

\pagebreak

\subsubsection{UC: Session vorbereiten}
\begin{tabularx}{\linewidth}{|l|X|}
	\hline
	Use Case Nr. 04			& \textbf{Moderator Session Vorbereiten} \\ \hline
	Erläuterungen			& Bevor ein Moderator eine Session mit den Teilnehmer anfangen 
	                          kann, muss er zuerst vorbereiten. Das heißt: er muss das ViVi einbetten und segmentieren, Verständnisfragen vorbereiten und TAN für die Session erzeugen. \\ \hline
	Systemgrenzen (Scope)	& Gesamtsystem. \\ \hline
	Ebene					& Hauptfunktion. \\ \hline
	Vorbedingung			& Die Web-App ist betriebsbereit. Der Benutzer hat 
	                          Moderator-Rechte. \\ \hline
	Mindestgarantie			& Im Fehlerfall wird keine Session erstellt.\\ \hline
	Erfolgsfall  			& Das segmentierte Vision-Video sowie eine sichere TAN sind 
	                          bereitgestellt für eine ViViPlayer-Session. \\ \hline
	Stakeholder				& Systembediener - will eine produktive Session mit den 
	                          Teilnehmer via ViViPlayer. \\
							& Herr Jianwei Shi (Systembesitzer) - möchte, dass alle Sessions 
							  in der Web-App erfolgreich durchgeführt werden. \\ \hline
	Hauptakteur				& Systembediener mit Moderator-Rolle. \\ \hline
	Auslöser				& Der Benutzer wählt die Möglichkeit, eine neue Session zu 
	                          erstellen, von der Moderator-Hauptseite. \\ \hline	
	Hauptszenario			& 1. Der Benutzer wählt die Möglichkeit, eine neue Session zu 
	                          erstellen, von der Moderator-Hauptseite. \\ 
							& 2. Das System zeigt die ViVi-Auswählen-Oberfläche an. \\
							& 3. Der Benutzer wählt ein ViVi und bestätigt. (UC 3.2.5) \\ 
							& 4. Das System zeigt die ViVi-Segmentieren-Oberfläche. \\ 
							& 5. Der Benutzer segmentiert das von ihm ausgewählte Video und 
							  bestätigt. (UC 3.2.6) \\
							& 6. Das System zeigt die Frage-Vorbereiten-Oberfläche. \\
							& 7. Der Benutzer schreibt die Multiple-Choice-Fragen für die 
							  Teilnehmer sowie den Typ der Frage und bestätigt. (UC 3.2.7) \\
							& 8. Das System zeigt die TAN-Oberfläche. \\
							& 9. Der Benutzer erzeugt eine neue TAN für die Session und 
							  bestätigt. (UC 3.2.8) \\ 
							& 10. Das System zeigt die ViVi-Überblick-Oberfläche. \\ 
							& 11. Der Benutzer prüft noch mal alle Vorbereitungschritte und 
							  erstellt neue Session. \\
							& 12. Das System zeigt die ViViPlayer-Oberfläche. \\ \hline
	Erweiterungen			& 4a. Falls das Video nicht eingebettet werden kann, wird eine 
	                          Fehlermeldung gezeigt. Das System geht zurück zu Schritt 2. \\
							& 6a. Falls die Segmentierung nicht gültig ist, wird eine 
							  Fehlermeldung ausgegeben. Das System geht zurück zu Schritt 4.\\
							& 8a. Falls es Fehler bei der Eingabe der Fragen, wird 
							  eine Fehlermeldung ausgegeben. Das System geht zurück zu Schritt 4. \\
							& 10a. Wenn die TAN nicht sicher genug ist, wird eine 
							  Fehlermeldung ausgegeben. Das System geht zurück zu Schritt 8. 
							  \\ \hline
	Priorität				& hoch \\ \hline
	Verwendungshäufigkeit	& häufig \\ \hline
\end{tabularx}
\\[0.5cm]
\textbf{Erläuterungen und Details}
\begin{itemize}
	\item ViVi: die Abkürzung von dem Begriff ``Vision-Video''.
	\item ViVi-Auswählen-Oberfläche: Eine Menge von ViVis werden hier angezeigt, damit der Benutzer eines davon auswählen kann.
	\item ViVi-Segmentieren-Oberfläche: da gibt es ein Video-Fenster mit Controller, eine Liste von Segmenten mit ihrem Namen und Zeitstempeln sowie ein Button zum Hinzufügen der neuen Segmente. 
	\item Frage-Vorbereiten-Oberfläche: es gibt eine Liste von erstellten Fragen
	sowie ein Fenster zum Bearbeitung einer neuen Frage.
	\item TAN-Oberfläche: man kann entweder manuell oder automatisch TAN für die Session hier erzeugen.
	\item ViVi-Überblick-Oberfläche: hier kann der Benutzer seine schon eingegebene Konfiguration nachschauen, um Fehler zu checken.
	\item Für die ViVi-Auswählen-, ViVi-Segmentieren-, TAN- und ViVi-Überblick-Oberfläche gibt es möglich ein ``Zurück''-Button, damit man nach Wunsch modifizieren kann.
\end{itemize}
\pagebreak

\subsubsection{UC: ViVi auswählen}
\begin{tabularx}{\linewidth}{|l|X|}
	\hline
	Use Case Nr. 05			& \textbf{Moderator Vision-Video Auswählen} \\ \hline
	Erläuterungen			& Vision-Video auswählen ist der erste Schritt zur Vorbereitung 
							  einer ViViPlayer-Session. \\ \hline
	Systemgrenzen (Scope)	& Gesamtsystem. \\ \hline
	Ebene					& Hauptebene \\ \hline
	Vorbedingung			& Die Web-App ist betriebsbereit. Der Benutzer hat 
							  Moderator-Rechte. \\ \hline
	Mindestgarantie			& Im Fehlerfall wird kein Video ausgewählt und somit keine 
							  Session.\\ \hline
	Erfolgsfall 			& Der Benutzer konnte sein gewünschtes Video problemlos auswählen. 
							  \\ \hline
	Stakeholder				& Systembediener - will eine produktive Session mit den Teilnehmer 
							  via ViViPlayer. \\
							& Herr Jianwei Shi (Systembesitzer) - möchte alle Sessions in der 
							  Web-App erfolgreich durchgeführt werden. \\ \hline
	Hauptakteur				& Systembediener mit Moderator-Rolle. \\ \hline
	Auslöser				& Der Benutzer wählt die Möglichkeit, eine neue Session zu 
							  erstellen, von der Moderator-Hauptseite. \\ \hline	
	Hauptszenario			& 1. Der Benutzer wählt die Möglichkeit, eine neue Session zu 
	                          erstellen, von der Moderator-Hauptseite. \\
							& 2. Das System zeigt die ViVi-Auswählen-Oberfläche an. \\
							& 3. Der Benutzer kann ein ViVi vom Server-Dateisystem auswählen 
							  und klickt auf dem ``Weiter''-Button, wenn er fertig ist. \\
							& 4. Das System zeigt die ViVi-Segmentieren-Oberfläche an. \\ \hline
	Erweiterungen			& 3a. Alternativ kann das Video via Pfad eingebettet werden. \\ 
							& 4a. Falls das Video nicht ausgewählt werden kann, wird eine 
							  Fehlermeldung gezeigt. Das System \\ \hline
	Priorität				& sehr hoch \\ \hline
	Verwendungshäufigkeit	& häufig \\ \hline
\end{tabularx}
\\[0.5cm]
\pagebreak

\subsubsection{UC: ViVi segmentieren}
\begin{tabularx}{\linewidth}{|l|X|}
	\hline
	Use Case Nr. 06			& \textbf{Moderator Vision-Video Segmentieren} \\ \hline
	Erläuterungen			& Vision-Video segmentieren ist der zweite Schritt zur 
							  Vorbereitung einer ViViPlayer-Session. \\ \hline
	Systemgrenzen (Scope)	& Applikation. \\ \hline
	Ebene					& Hauptebene \\ \hline
	Vorbedingung			& Die Web-App ist betriebsbereit. Der Benutzer hat Moderator-Rechten \\ \hline
	Mindestgarantie			& Im Fehlerfall wird das Video nicht segmentiert. \\ \hline
	Erfolgsgarantie			& Das ViVi wird erfolgreich segmentiert. \\ \hline
	Stakeholder				& Der Benutzer - will eine produktive Session mit den Teilnehmer via
							  ViViPlayer. \\
							& Herr Jianwei Shi - möchte alle Sessions in der Web-App erfolgreich durchgeführt
							  werden. \\ \hline
	Hauptakteur				& Der Benutzer \\ \hline
	Auslöser				& Der Benutzer hat schon ein ViVi ausgewählt und klick auf ``Weiter''-Button. 
							  \\ \hline	
	Hauptszenario			& 1. Das System zeigt ViVi-Segmentieren-Oberfläche. \\
							& 2. Von dem Video-Fenster spielt der Benutzer das Video ab, bis zu seinem
							  gewünschten Zeitstempel für ein neues Segment, danach benennt er das Segment nach 
							  seinem Wunsch und klickt ``Hinzufügen''. \\
							& 3. Das System fügt das neue Segment in der Segmente-Liste hinzu. \\
							& 4. Der Benutzer klickt auf ``Weiter''-Button. \\
							& 5. Das System zeigt Verständnisfrage-Vorbereiten-Oberfläche   
							  an. \\ \hline
	Erweiterung				& 2a. Alternativ kann der Benutzer die automatische 
							  Segmentierung nutzen. \\
							& 3a. Wenn der Benutzer noch nicht fertig mit dem Segmentierung 
							  ist, geht der Benutzer zurück zu Schritt 2. \\ 
							& 3b. Falls der Name eines Segments leer oder eine Zeitstempel 
							  nicht gültig ist, wird eine Fehlermeldung ausgegeben. Kein neues Segment wird hinzugefügt. \\ \hline
	Priorität				& sehr hoch \\ \hline
	Verwendungshäufigkeit	& häufig \\ \hline
\end{tabularx}
\\[0.5cm]
\textbf{Erläuterungen und Details}
\begin{itemize}
	\item Shot: Ein anonymer Begriff für Segment.
\end{itemize}
\pagebreak

\subsubsection{UC: Um-/Verstänisfragen erstellen}
\begin{tabularx}{\linewidth}{|l|X|}
	\hline
	Use Case Nr. 07			& \textbf{Moderator Fragen Erstellen} \\ \hline
	Erläuterungen			& Fragen erstellen ist der dritte Schritt zur Vorbereitung einer 
							  ViViPlayer-Session. \\ \hline
	Systemgrenzen (Scope)	& Gesamtsystem. \\ \hline
	Ebene					& Hauptfunktion. \\ \hline
	Vorbedingung			& Die Web-App ist betriebsbereit. Der Benutzer hat Moderator
							  -Rechte und schon ein ViVi ausgewählt (UC 3.2.5) und segmentiert (UC 3.2.6). \\ \hline
	Mindestgarantie			& Im Fehlerfall wird die Fragen nicht erstellt. 
							  Das ViVi wird nicht gelöscht. \\ \hline
	Erfolgsfall    			& Eine Liste von Verständnisfragen ist erfolgreich hinzugefürgt.
							  \\ \hline
	Stakeholder				& Systembediener - will eine produktive Session mit anderen 
	                          Benutzer via ViViPlayer. \\
							& Herr Jianwei Shi (Systembesitzer) - möchte, dass alle Sessions 
							  in der Web-App erfolgreich durchgeführt werden. \\ \hline
	Hauptakteur				& Systembediener mit Moderator-Rolle. \\ \hline
	Auslöser				& Der Benutzer befindet sich auf der ViVi-Segmentieren-Seite und 
							  klickt auf ``Weiter''. \\ \hline	
	Hauptszenario			& 1. Der Benutzer befindet sich auf der ViVi-Segmentieren-Seite 
	                          und klickt auf ``Weiter''. \\  
							& 2. Das System zeigt die Frage-Vorbereiten-Oberfläche. \\
							& 3. Der Benutzer tippt neue Frage, die Auswahlen für die 
							  Frage, sowie den Zeitstempel und den Typ der Frage ein und 
							  bestätigt. \\ 
							& 4. Das System fügt die neue Frage in der Liste hinzu. \\ 
							& 5. Der Benutzer klickt auf dem ``Weiter''-Button. \\ 
							& 6. Das System zeigt die TAN-Oberfläche an. \\ \hline
	Erweiterungen			& 3a. Wenn der Benutzer noch nicht fertig mit der 
							  Fragen-Vorbereitung ist, geht er zurück zu Schritt 3. \\
							& 3b. Wenn die Frage leer ist oder es keine Auswahl für die 
							  Frage gibt oder der Zeitstempel nicht gültig ist, wird eine 
							  Fehlermeldung angezeigt. Das System geht zurück zu Schritt 3.
							  \\ \hline
	Priorität				& mittel \\ \hline
	Verwendungshäufigkeit	& normal \\ \hline
\end{tabularx}
\pagebreak

\subsubsection{UC: ViVi navigieren}
\begin{tabularx}{\linewidth}{|l|X|}
	\hline
	Use Case Nr. 08			& \textbf{ViVi navigieren} \\ \hline
	Erläuterungen			& Als Moderator kann der Benutzer das ViVi navigieren. \\ \hline
	Systemgrenzen (Scope)	& Gesamtsystem. \\ \hline
	Ebene					& Hauptfunktion. \\ \hline
	Vorbedingung			& Die Web-App ist betriebsbereit. Der Benutzer hat Moderator-
							  Rechte und hat schon eine ViViPlayer-Session erstellt (UC 5, UC 6, UC 7). Der ist in einer Session mit anderen Benutzern, die entweder Moderator oder Teilnehmer sind. \\ \hline
	Mindestgarantie			& Im Fehlerfall wird das ViVi beim Moderator oder bei Teilnehmer 
							  nicht navigiert. Das ViVi bleibt unverändert. \\ \hline
	Erfolgsfall 			& Das ViVi wurde für alle Benutzer in der Session navigiert. 
							  \\ \hline
	Stakeholder				& Systembediener (Moderator) - will eine produktive Session mit den 
							  Teilnehmer via ViViPlayer. \\
							& Systembesitzer (Systemadministrator) - möchte alle Sessions in der 
							  Web-App erfolgreich durchgeführt werden. \\ \hline
	Hauptakteur				& Systembediener (Moderator). \\ \hline
	Auslöser				& Der Benutzer navigiert das Video zu seinem gewünschten 
							  Zeitpunkt. \\ \hline	
	Hauptszenario			& 1. Der Benutzer navigiert das Video zu seinem gewünschten 
							  Zeitpunkt. \\
							& 2. Das System erkennt die Navigation von dem Benutzer, und 
							  schickt ein Signal zu allen anderen Benutzer. \\ 
							& 3. Das ViVi wird für alle Benutzer in der Session navigiert. 
							  \\ \hline
	Erweiterungen			&  \\ \hline
	Priorität				& sehr hoch \\ \hline
	Verwendungshäufigkeit	& sehr häufig \\ \hline
\end{tabularx}
\pagebreak

\subsubsection{UC: Um-/Verstänisfragen in Session stellen}
\begin{tabularx}{\linewidth}{|l|X|}
	\hline
	Use Case Nr. 09			& \textbf{Um-/Verständnisfrage in Session stellen} \\ \hline
	Erläuterungen			& Als Moderator kann der Benutzer mehrere Um-/Verständnisfragen 
							  während der Session erstellen, um Feedbacks von Teilnehmer zu bekommen. \\ \hline
	Systemgrenzen (Scope)	& Gesamtsystem. \\ \hline
	Ebene					& Hauptfunktion. \\ \hline
	Vorbedingung			& Die Web-App ist betriebsbereit. Der Benutzer hat Moderator-
							  Rechten. Er ist in einer Session mit anderen Benutzern. \\ \hline
	Mindestgarantie			& Keine Umfrage wird erstellt. \\ \hline
	Erfolgsfall 			& Die Um-/Verständnisfrage, die 30 Sekunde dauert, wurde erstellt, 
							  alle Teilnehmer konnten die sehen und interagieren. \\ \hline
	Stakeholder				& Systembediener (Moderator) - will eine produktive Session mit den 
							  Teilnehmer via ViViPlayer. \\
							& Systembesizer (Herr Jianwei Shi) - möchte alle Sessions in der 
							  Web-App erfolgreich durchgeführt werden. \\ \hline
	Hauptakteur				& Systembediener (Moderator). \\ \hline
	Auslöser				& Der Benutzer klickt auf ``Frage''-Tab. \\ \hline	
	Hauptszenario			& 1. Der Benutzer klickt auf ``Frage''-Tab. \\
							& 2. Das System zeigt ein neues Eingabe-Feld an. \\ 
							& 3. Der Benutzer gibt den Frage-Texte und den Typ der Frage (Um- 
							  oder Verständnisfrage) für die ein und bestätigt. \\
							& 4. Das System zeigt die Um-/Verständnisfrage in 30 Sekunden für 
							  alle Teilnehmer an. \\ \hline
	Erweiterungen			& 4a. WENN es keine Frage-Texte oder keine Auswahl für die 
							  Um-/Verständnisfrage gibt, DANN zeigt das System eine Fehlermeldung an. Zurück zu Schritt 2. \\ \hline
	Priorität				& mittel \\ \hline
	Verwendungshäufigkeit	& weniger häufig \\ \hline
\end{tabularx}
\pagebreak

\subsubsection{UC: User-Story schreiben}
\begin{tabularx}{\linewidth}{|l|X|}
	\hline
	Use Case Nr. 10			& \textbf{User-Story schreiben} \\ \hline
	Erläuterungen			& Systembediener möchte eine User Story schreiben. \\ \hline
	Systemgrenzen (Scope)	& User-Story-System. \\ \hline
	Ebene					& Hauptfunktion. \\ \hline
	Vorbedingung			& Das System ist betriebsbereit. Der Benutzer hat Moderator- oder 
							  Teilnehmer-Rolle, der befindet sich in einer ViViPlayer-Session. \\ \hline
	Mindestgarantie			& Im Fehlerfall erhält der Benutzer eine Fehlermeldung und es 
							  werden keine Daten in dem Server abgespeichert. \\ \hline
	Erfolgsgarantie			& Der Benutzer hat eine User Story verfasst und der Server hat
							  diese abgespeichert. Die User Story wurde einem Shot
							  zugeordnet. \\ \hline
	Stakeholder				& Systembediener (Moderator/Teilnehmer) - möchte User Stories 
							  einfach verfassen können.\\ 
                            & Entwickler - möchten User Stories für ihr Projekt haben. \\ 
                              \hline
	Hauptakteur				& Systembediener (Moderator/Teilnehmer). \\ \hline
	Auslöser				& Der Benutzer klickt auf das Textfeld für User Story und gibt 
							  seine User Story ein, dann bestätigt. \\ \hline	
	Hauptszenario			& 1. Der Benutzer klickt auf das Textfeld für User Story und gibt 
							  seine User Story ein, dann bestätigt. \\
							& 2. Das System verarbeitet diese Anfrage und speichert die User 
							  Story ab. Zudem wird sie dem jeweiligen Shot zugeordnet. \\
							& 3. Das System zeigt die neue User Story den anderen Nutzern an.\\
							& 4. Der Nutzer kann eine weitere User Story verfassen. \\ \hline
	Erweiterungen			& 2a. WENN ein Fehler auftritt DANN wird der Benutzer 
	                          benachrichtigt und er kann zurück zu Schritt 1 gehen. Es werden keine Daten in der Datenbank abgespeichert. \\ \hline
	Priorität				& hoch \\ \hline
	Verwendungshäufigkeit	& häufig \\ \hline
\end{tabularx}

\\[0.5cm]
\pagebreak

\subsubsection{UC: In Session kommentieren}
\begin{tabularx}{\linewidth}{|l|X|}
	\hline
	Use Case Nr. 11			& \textbf{In Session kommentieren} \\ \hline
	Erläuterungen			& Der Benutzer möchte einen Kommentar/Satz verfassen. \\ \hline
	Systemgrenzen (Scope)	& Kommentar-System. \\ \hline
	Ebene					& Hauptfunktion. \\ \hline
	Vorbedingung			& Das System ist betriebsbereit. Der Benutzer hat Moderator- oder 
							  Teilnehmer-Rechte, der befindet sich auf der ViViPlayer-Seite der Web-App. \\ \hline
	Mindestgarantie			& Der Benutzer erhält eine Fehlermeldung und es werden keine Daten 
							  in der Datenbank abgespeichert. Die Daten von anderen Benutzer werden abgespeichert. \\ \hline
	Erfolgsfall				& Der Benutzer hat eine Nachricht verfasst und der Server hat 
							  diese abgespeichert. Die Nachricht wurde einem Segment zugeordnet.\\ \hline
	Stakeholder				& Systembediener (Moderator/Teilnehmer) - möchte Kommentare einfach 
							  verfassen oder Feedback geben können.\\ 
                            & Entwickler - möchten Kommentare für ihr Projekt haben und 
                              verfassen können. \\ \hline
	Hauptakteur				& Systembediener (Moderator/Teilnehmer) \\ \hline
	Auslöser				& Der Benutzer wählt den Kommentar-Modus aus. \\ \hline	
	Hauptszenario			& 1. Der Benutzer wählt den Kommentar-Modus aus. \\
                            & 2. Der Benutzer gibt seinen Kommentar ein und bestätigt, wenn er 
                              fertig ist\\
							& 3. Das System verarbeitet diese Anfrage und speichert den 
							  Kommentar ab. Zudem wird er dem jeweiligen Segment zugeordnet. \\
							& 4. Das System zeigt den neuen Kommentar den anderen Nutzern an.\\
							& 5. Der Nutzer kann einen weiteren Kommentar verfassen. \\ \hline
	Erweiterungen			& 3a. WENN ein Fehler auftritt, DANN wird der Benutzer 
							  benachrichtigt und er kann den Kommentar nochmal eingeben. Es werden keine Daten in der Datenbank abgespeichert.\\ \hline
	Priorität				& hoch \\ \hline
	Verwendungshäufigkeit	& sehr häufig \\ \hline
\end{tabularx}

\pagebreak

\subsubsection{UC: Log-Dateien exportieren}
\begin{tabularx}{\linewidth}{|l|X|}
	\hline
	Use Case Nr. 12			& \textbf{Log-Dateien exportieren} \\ \hline
	Erläuterungen			& Nach der Beendigung der ViViPlayer-Session sollte einige Dateien
							  als Ergebnissicherung exportiert werden.\\ \hline
	Systemgrenzen (Scope)	& Gesamtsystem. \\ \hline
	Ebene					& Hauptfunktion. \\ \hline
	Vorbedingung			& Das System ist betriebsbereit. Der Benutzer befindet sich auf der 
							  ViViPlayer-Seite der Web-App. \\ \hline
	Mindestgarantie			& Die Log-Dateien werden mit einer Fehlermeldung exportiert. Die 
							  Daten von den Benutzer, die nicht Fehler haben, werden exportiert. Das Dateisystem vom Server funktioniert einwandfrei.\\ \hline
	Erfolgsfall				& Die exportierten Dateien wurden im Server abgespeichert. Der 
							  Benutzer konnte solche Dateien problemlos herunterladen. \\ \hline
	Stakeholder				& Systembediener (Moderator) - möchte Log-Dateien exportieren 
							  können.\\ 
							& Entwickler - möchten die Log-Dateien zu Trello- oder Jira-Board 
							  importieren können. \\ \hline
	Hauptakteur				& Systembediener (Moderator) \\ \hline
	Auslöser				& Der Benutzer klickt auf das "Session Beenden"-Button. \\ \hline	
	Hauptszenario			& 1. Der Benutzer klickt auf das "Session Beenden"-Button. \\
							& 2. Das System sammelt alle User Stories und Kommentare von den 
							  Benutzern sowie Screenshot des ViVis und speichert die im Server 
							  ab. \\
							& 3. Der Benutzer klickt auf das ``Exportieren''-Button. \\
							& 4. Das System erlaubt dem Benutzer, die Dateien herunterzuladen. \\
							& 5. Der Benutzer speichert die Dateien in seinem System ab. 
							  \\ \hline
	Erweiterungen			& 2a. WENN ein Fehler auftritt, DANN wird die Log-Dateien mit 
							  Fehlermeldung im Server-Dateisystem abgespeichert. Der Benutzer bekommt dann keine URL zum Herunterladen. \\ \hline
	Priorität				& sehr hoch \\ \hline
	Verwendungshäufigkeit	& regelmäßig \\ \hline
\end{tabularx}


\pagebreak

\subsubsection{UC: An Um-/Verständnisfrage teilnehmen}
\begin{tabularx}{\linewidth}{|l|X|}
	\hline
	Use Case Nr. 13			& \textbf{An Um-/Verständnisfragen teilnehmen} \\ \hline
	Erläuterungen			& Während der Session haben alle Teilnehmr die Möglichkeit, an einer 
							  Um-/Verständnisfrage des Moderators teilzunehmen. \\ \hline
	Systemgrenzen (Scope)	& Gesamtsystem. \\ \hline
	Ebene					& Hauptfunktion. \\ \hline
	Vorbedingung			& Das System ist betriebsbereit. Der Benutzer befindet sich auf der 
							  ViViPlayer-Seite der Web-App. Da stellt der Moderator eine Um-/Verständnisfrage für alle Teilnehmer und das System zeigt die auf das Fenster für Um-/Verständnisfrage. \\ \hline
	Mindestgarantie			& Im Fehlerfall wird die Antwort des Benutzers nicht gezählt und 
							  nicht sichtbar zu den anderen sein. \\ \hline
	Erfolgsfall				& Der Moderator kann die Antwort vom Benutzer sehen oder die Antwort 
							  wird für das Balkendiagramm gerechnet. \\ \hline
	Stakeholder				& Systembediener (Teilnehmer) - möchte seine Meinung durch die 
							  Antwort für Um-/Verständnisfragen mitteilen. \\ 
							& Systembediener (Moderator) - möchte die Meinungen von Teilnehmer 
							  sammeln, damit die Lösung immer optimal ist. \\ \hline
	Hauptakteur				& Systembediener (Teilnehmer) \\ \hline
	Auslöser				& Der Benutzer klickt auf die Antwort der Um-/Verständnisfrage, die 
							  ihm gefällt und dann klickt auf das "Senden"-Button. \\ \hline	
	Hauptszenario			& 1. Der Benutzer klickt auf eine Antwort der Um-/Verständnisfrage, 
							  die ihm gefällt und dann klickt auf das "Senden"-Button. \\
							& 2. Das System sammelt die Antwort von allen Benutzer. Nach 30 
							  Sekunden zeigt es das Ergebnis von der Um-/Verständnisfragen im 
							  Form eines Balkendiagramm. \\ \hline
	Erweiterungen			& 2a. WENN der Benutzer keine Antwort eingibt, DANN zählt das System 
							  seine leere Antwort auch nicht. \\ \hline
	Priorität				& hoch \\ \hline
	Verwendungshäufigkeit	& regelmäßig \\ \hline
\end{tabularx}


\pagebreak
\section{Probleme und Risiken}
\begin{enumerate}
	\item
    \textbf{WENN} es zu Verzögerungen im Projekt kommt \textbf{DANN} reicht die Zeit nicht aus, um nicht wesentliche Anforderungen zu entwickeln. Kunde wird am Ende unzufrieden sein.
    \linebreak
    \linebreak
    \textbf{Wahrscheinlichkeit:}  Da dieses Team zum ersten Mal zusammengearbeitet hat, ist es sehr schwer abzuschätzen, wie viel das Team in einem Sprint erreichen kann.
    \linebreak
    \linebreak
    \textbf{Abhilfe:} Das Team muss sich auf die obligatorischen Pflichtanforderungen konzentrieren, bevor es über zusätzliche Anforderungen nachdenkt.
    \linebreak

    \item
    \textbf{WENN} die Technologie mit dem bereitgestellten Produktionsserver nicht richtig funktioniert \textbf{DANN} sind Änderungen am Server und/oder der Software erforderlich, um die Kompatibilität zu gewährleisten. Dies kann später im Projektzyklus zu großen Verzögerungen führen.
    \linebreak
    \linebreak
    \textbf{Wahrscheinlichkeit:} Da der Server von der IT-Abteilung der Universität bereitgestellt wird und wir ein empfohlenes Framework (Django) verwenden, ist die Wahrscheinlichkeit einer Inkompatibilität gering.
    \linebreak
    \linebreak
    \textbf{Abhilfe:} Die Website sollte so früh wie möglich auf dem Produktionsserver getestet, um eventuelle Kompatibilitätsprobleme abzuschätzen.
    \linebreak
    
    \item
    \textbf{WENN} nur 1 Entwickler allein für den Kerncode verantwortlich ist \textbf{DANN} kann es zu Problemen kommen, wenn der Entwickler krankheits- oder aus anderen persönlichen Gründen arbeitsunfähig ist. Dies kann zu großen Verzögerungen im Projekt führen.
    \linebreak
    \linebreak
    \textbf{Wahrscheinlichkeit:} Es besteht immer die Möglichkeit, dass ein Entwickler seine Arbeit nicht fortsetzen kann.
    \linebreak
    \linebreak
    \textbf{Abhilfe:} Alle Entwickler sollten über ein funktionierendes Verständnis aller Aspekte des Quellcodes verfügen.
    \linebreak

    \item
    \textbf{WENN} es Verzögerungen bei der Synchronisierung von Videos gibt \textbf{DANN} können diese Verzögerungen dazu führen, dass sich Benutzer an einem falschen Ort befinden oder eine ruckartige Wiedergabe erleben.
    \linebreak
    \linebreak
    \textbf{Wahrscheinlichkeit:} Aufgrund normaler Schwankungen in der Verbindungsgeschwindigkeit und -verzögerung ist es sehr wahrscheinlich, dass Benutzer ein gewisses Maß an Synchronisierungsproblemen haben.
    \linebreak
    \linebreak
    \textbf{Abhilfe:}  Kleine Verzögerungen können ignoriert werden, um eine ruckelige Wiedergabe zu reduzieren. Die Videoqualität kann für Benutzer mit langsameren Verbindungen reduziert werden. Einschränkungen der Videogröße können implementiert werden.
    \linebreak
    
    \item
    \textbf{WENN} Benutzer einen inkompatiblen Browser verwenden \textbf{DANN} funktionieren möglicherweise bestimmte Funktionen nicht wie beabsichtigt.
    \linebreak
    \linebreak
    \textbf{Wahrscheinlichkeit:} Aufgrund der Vielzahl von Browsern, die unter anderem auf mobilen Geräten verwendet werden, ist es möglich, dass viele Browser nicht alle HTML 5-Standards oder -Funktionen unterstützen.
    \linebreak
    \linebreak
    \textbf{Abhilfe:} Die Website kann auf den gängigsten Webbrowsern der Welt getestet werden. Dies wird die Mehrheit der Benutzer abdecken. Außerdem können den Benutzern Browser-Empfehlungen präsentiert werden, um die Kompatibilität sicherzustellen.
    \linebreak


\end{enumerate}


